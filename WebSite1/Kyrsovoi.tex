\documentclass[14pt,a4paper]{extreport}
\usepackage[top=2cm, left=3cm, bottom=2cm, right=1cm]{geometry}
\usepackage[utf8x]{inputenc} % Включаем поддержку UTF8
\usepackage[russian]{babel} % Пакет поддержки русского языка
\usepackage{lscape}
\usepackage{fancyhdr}
\usepackage{textcase}
\graphicspath{{pictures/}}
\DeclareGraphicsExtensions{.pdf,.png,.jpg}
\usepackage{caption}
\usepackage{graphicx}
\usepackage{caption}
\usepackage{refstyle}

\title{}
\author{}

\begin{document}

%----------ТИТУЛЬНЫЙ-ЛИСТ--------------
	\center
	Министерство образования Республики Беларусь\\
	Учреждение образования «Белорусский государственный университет информатики и радиоэлектроники»
	\vspace*{2cm}
	\endcenter
	\raggedright
	Факультет компьютерных систем и сетей\\
	\medskip
	Кафедра программного обеспечения информационных технологий\\
	\medskip
	Дисциплина:  Компьютерные системы и сети (КСиС)
	\vspace*{2cm}
	\center
	ПОЯСНИТЕЛЬНАЯ ЗАПИСКА\\
	к курсовому проекту\\
	на тему\\
	\medskip
	Сайт про F1 c чтением RSS канадов\\
	\medskip
	БГУИР КП  1-40 01 0 22 ПЗ
	\vspace*{4cm}
	\endcenter
	\raggedright
	\hspace*{7.94cm} Студент: гр. 351002 Сергушов Н.О.\\
	\bigskip
	\hspace*{7.94cm}РуководитЦель: асс. Третьяков Ф.И.\\
	\center
	\vspace*{2cm}
	Ми	нск 2015
	\pagestyle{empty}
%-------ЛИСТ-ЗАДАНИЯ--------------
	\newpage
	\center
	Учреждение образования\\
	\medskip
	«Белорусский государственный университет информатики и радиоэлектроники»\\
	\medskip
	Факультет компьютерных систем и сетей\\
	\medskip
	\endcenter
	\raggedright
	\hspace*{9.53cm}УТВЕРЖДАЮ\\
	\hspace*{9.53cm}Заведующий кафедрой ПОИТ\\
	\hspace*{9.53cm}\underline{\hspace{6cm}} \\
	\hspace*{11cm}\small (подпись) \normalsize\\
	\hspace*{9.53cm}\underline{\hspace{5cm}}2015 г.\\
	\medskip
	\center
	ЗАДАНИЕ\\
	по курсовому проектированию\\
	\medskip
	\endcenter
	\raggedright
	Студенту \underline{ Сергушову Никите Олеговичу }\\
	\begin{enumerate}
	\item Тема работы \underline{Сайт формулы 1 с RSS фидами}\\ 
	\item Срок сдачи студентом законченной работы \underline{DD.MM.YYYY}
	\item Исходные данные к работе \underline{Среда разработки Sublime text . }
	\item Содержание расчётно-пояснительной записки (перечень вопросов, которые подлежат разработке)\\
	\underline{\hspace*{16cm}}\hspace*{-16cm}Введение. 1. Анализ литературных источников. 2. Постановка задачи\\
	\underline{\hspace*{16cm}}\hspace*{-16cm}3.Разработка программного средства. 4. Руководство по \\
	\underline{\hspace*{16cm}}\hspace*{-16cm}использованию веб-сайта. Заключение. Приложения.
	\item Перечень графического материала (с точным обозначением обязательных чертежей и графиков)\\
	\underline{1. Схема алгоритма}
	\item Консультант по курсовой работе\\
	\underline{Третьяков Ф.И.}  
	\item Дата выдачи задания \underline{DD.MM.YYYY}
	\item Календарный график работы над проектом на весь период проектирования (с обозначением сроков выполнения и процентом от общего объёма работы):\\
	\underline{\hspace*{16cm}}\hspace*{-16cm}раздел 1 к DD.MM.YYYY – 15 \% готовности работы;\\  
	\underline{\hspace*{16cm}}\hspace*{-16cm}разделы 2, 3 к DD.MM.YYYY – 30 \% готовности работы;\\ 
	\underline{\hspace*{16cm}}\hspace*{-16cm}раздел 4 к DD.MM.YYYY – 60 \% готовности работы;\\
	\underline{\hspace*{16cm}}\hspace*{-16cm}раздел 5, 6 к DD.MM.YYYY  –  90 \% готовности работы;\\
	\underline{\hspace*{16cm}}\hspace*{-16cm}оформление пояснительной записки и графического материала к\\
	\underline{\hspace*{16cm}}\hspace*{-16cm}DD.MM.YYYY – 100 \% готовности работы.\\
	\underline{\hspace*{16cm}}\hspace*{-16cm}Защита курсового проекта с DD по DD декабря YYYY г.\\
	\end{enumerate}
	\hspace*{7cm}РУКОВОДИТЕЛЬ\underline{\hspace*{6cm}}\hspace*{-3.9cm}Третьяков Ф.И.\\
	\hspace*{11.5cm}\small (подпись) \normalsize\\
	\bigskip
	Задание принял к исполнению \underline{\hspace*{10.5cm}}\hspace*{-8cm}Пасечный А.В. DD.MM.YYYYг.\\
	\hspace*{7cm}\small (дата и подпись студента) \normalsize\\
	%-------СОДЕРЖАНИЕ--------------
	\newpage
	\pagestyle{plain}
	%\renewcommand{\headrulewidth}{0px}
	%\fancypagestyle{plain}{\cfoot{}\rfoot{\thepage}}
	
	\renewcommand\contentsname{\center\normalsize \textbf{СОДЕРЖАНИЕ} \endcenter}
	\tableofcontents
	\endcenter
	



	%-----ВВЕДЕНИЕ-----
	\newpage
	\addcontentsline{toc}{section}{ВВЕДЕНИЕ}
	\section*{\center\normalsize ВВЕДЕНИЕ \endcenter}
	\hspace{4ex}Rss технология весьма простая и эффективная. Чтобы начать ей пользоваться не нужно никаких усилий. Прежде всего, нужно осознать, как работает эта система. Это позволит понять и оценить всю пользу и ценность новостных лент.
Rss – это xml-формат данных, который позволяет следить за обновлениями сайтов в интернете. Несколько лет назад любой серьезный сайт предоставлял возможность подписки на свой RSS-канал. Сейчас наблюдается тенденция «устаревания», на сайтах перестали делать rss-ленты и привычную оранжевую иконку можно обнаружить все реже.   Но сам RSS от этого не стал хуже.

\hspace{4ex}На каждом сайте периодически публикуется какой-то контент — статьи, новости, комментарии. Подписка на RSS-ленту новостей сайта позволяет узнавать об обновлениях на этом сайте максимально оперативно, без необходимости посещать и просматривать сам сайт. Это сродни просмотру заголовков утренней газеты. Зачем идти на сайт и просматривать его в поисках новостей, если можно просматривать анонсы этих статей у себя на компьютере, и в случае находки интересного материала пройти по ссылке на свежую статью, заголовок которой обязательно будет в RSS-ленте.  Мало того, что это сберегает кучу времени, так еще и уменьшает трафик, ибо вы идете на сайт только если вам действительно что-то интересно, наверняка у вас есть любимые сайты, которые вы периодически посещаете, читаете статьи и новости, следите за обновлениями.

\hspace{4ex}Ленты новостей обрабатываются специальными программами — RSS-агрегатами.  Агрегаты бывают двух видов — web-сервисы и программы на вашем компьютере. Какой из них выбрать — это дело вкуса. Онлайн-агрегаты по функциональности чем-то похожи на почтовые сервисы — вы создаете аккаунт в Сети на одном из сервисов, и все ваши ленты будут приходить в ящик на вашем аккаунте. Этот способ хорош тем, что вы всегда будете иметь доступ к вашим лентам, независимо от того где вы находитесь. Но у них есть недостаток, что все же нужно идти и проверять ящик. То есть, чтобы узнать о новостях или подписаться на новую ленту вам открывать в браузере страницу сервиса-агрегата.

\hspace{4ex}Поэтому есть еще более удобные — локальные RSS-агрегаты, которые могут быть встроенными в браузер, почтовик или быть отдельным приложением.  Если вы знаете что такое почтовая программа, то вы поймете преимущество новостных лент. Локальный агрегат позволяет следить за новостями гораздо эффективнее — вы получаете обновления на сайтах практически моментально. Локальные RSS-агрегаты сами «бегают» на сайт и, если там появилось что-то новенькое «приносят» вам заголовок или анонс статьи, а иногда и всю статью — это зависит от того какую ленту предоставляет сайт.

\hspace{4ex}Цель данной курсовой работы заключается в создании веб-сайта F1, который будет отображать новости интересующих RSS-каналов. 

\hspace{4ex}Данная пояснительная записка состоит из ряда разделов, которые охватывают как теоретические, так и практические аспекты разработки программного средства. 
\




	%-----1 АНАЛИЗ ЛИТЕРАТУРНЫХ ИСТОЧНИКОВ----
	
	\newpage
	\addcontentsline{toc}{section}{1 АНАЛИЗ ЛИТЕРАТУРНЫХ ИСТОЧНИКОВ}
	\addcontentsline{toc}{subsection}{1.1 Обзор аналогов}
	\addcontentsline{toc}{subsection}{1.2 RSS формат}
	\addcontentsline{toc}{subsubsection}{1.2.1 Структура RSS}
	\addcontentsline{toc}{subsection}{1.3 Конкуренция с форматом Atom}

	\item\section*{\normalsize\hspace{2ex}АНАЛИЗ ЛИТЕРАТУРНЫХ ИСТОЧНИКОВ}
	
	%--------------------------1.1 Обзор аналогов----------------------------------
	\textbf{1.1  Обзор аналогов}
	
\hspace {4ex}На данный момент из популярных браузеров возможность обработки rss ленты новостей реализована только Mozilla.

\hspace{4ex}Firefox умеет обрабатывать rss-ленту в xml-формате.    Если вы пользуетесь браузером Firefox, при попытке подписаться на RSS вам откроется страничка, где вы сможете выбрать онлайн-сервис новостных лент. Сам Mozilla Firefox поддерживает подписку на RSS только в виде закладок на новости.

\hspace{4ex}Браузер Google Сhrome также не умеет работать c RSS-каналами, но такой функционал можно в него добавить в виде расширений. Например, вот так выглядит расширение Slick RSS для Chrome.

\hspace{4ex}Онлайн-сервисы для подписки на RSS-каналы абсолютно бесплатны, и чем-то похожи на почтовые. Самые популярный это — Яндекс. Лента.  Для того чтобы воспользоваться сервисом нужно иметь Yandex аккаунт.   Если вы уже пользуетесь какими-либо сервисами Яндекса, то нужно лишь войти под своим логином и подписаться на ленту. После чего вы сможете читать новостную ленту через интерфейс Яндекс-почты.

	%--------------------------1.2 RSS формат----------------------------------
	\textbf{1.2  Rss формат}\par
           \hspace{4ex}На данный момент существует 7 различных форматов RSS. Как программисту, пишущему программу-агрегат, вам придется сражаться со всеми этими форматами. Ну а какой формат выбрать пользователю, публикующему свои новости в формате RSS?
           \center \includegraphics{RSS_ALL}\par
	\center Рисунок 1.1 – RSS форматы\par
\flushleft\parindent=1cm В 1970 году выдается патент на компьютерный манипулятор, без которого сейчас не может обойтись не один игрок. Речь идет о компьютерной мышке. Человека, получившего патент, звали Дуглас Энгельбарт. 1975 год стал годом проявления интереса к компьютерным играм со стороны общественности. Начиная с 1977 года, различные разработчики выпускают все больше и больше новых компьютерных игр, которые в последствии значительно ускорят развитие персональных компьютеров.\par
           
           %--------------------------1.2.1 Структура RSS----------------------------------
           \textbf{  1.2.1 Структура RSS}
\hspace{4ex}Скажем, вы захотели написать программу, которая считывает новости в формате RSS, чтобы, например, публиковать заголовки новостей на своем сайте, или чтобы создать портал новостей и так далее. Как выглядит RSS-файл? Все зависит от того, о какой версии RSS идет речь. Вот пример файла в формате RSS 0.91 :
           \center \includegraphics{RSS}\par   
           \center Рисунок 1.2 – RSS формат 0.91\par
           
	\flushleft\parindent=1cm Всё выглядит достаточно просто. Блок новостей (channel) состоит из заголовка, ссылки, данных о языке новостей и описания. После этого идет список самих новостей, где в каждом пункте указывается заголовок, ссылка и краткое описание новости.\par
Теперь давайте взглянем, как та же самая информация выглядит в формате RSS 1.0:\par'

	\includegraphics{RSSone}   
           \center Рисунок 1.3 – RSS формат 1.0\par
           
\flushleft\parindent=1cm Те, кто знаком с RDF, сразу узнают, что этот файл - RDF-документ, сохраненный в XML. Остальные, разберутся, что в файле представлена вся та же информация, что и в первом примере. Однако тут добавлена еще некоторая дополнительную информацию, как, например, авторство каждой новости, и дата публикации, которых нет в RSS 0.91.\par
Несмотря на то, что RSS 1.0 является смесью RDF и XML, структурно он схож с предыдущими версиями RSS - схож достаточно, чтобы рассматривать его как обычный XML-файл. Следовательно, можно написать одну программу, которая умеет извлекать информацию из обоих форматов: и из RSS 0.91 и из RSS 1.0. Однако есть все-таки некоторые различия, о которых программа должна знать:
\begin {enumerate} 
\item Корневым элементом в RSS 1.0 является rdf:RDF, а не rss. Вам либо придется явно обрабатывать оба этих элемента, либо просто игнорировать их и слепо извлекать только ту информацию, которая вам нужна.
\item В RSS 1.0 используются пространства имен (namespaces). Пространство имен для RSS 1.0 выглядит так http://purl.org/rss/1.0/. И это пространство имен принимается по умолчанию. Кроме того в файле используются пространства имен http://www.w3.org/1999/02/22-rdf-syntax-ns для элементов, специфичных для RDF (мы их тоже можем игнорировать), и http://purl.org/dc/elements/1.1/ (Dublin Core) для дополнительных метаданных об авторах статей и датах публикаций.
Можно пойти двумя путями: если XML-парсер не понимает пространства имен, вы можете просто считать, что в файле используются элементы с префиксами и слепо искать в них элементы items и dc:creator. Такой способ сработает в большинстве случаев, так как в новостях формата RSS 1.0 чаще всего используется только пространство имен, принятое по умолчанию, и пространство имён Dublin Core. Конечно, данный способ - не элегантен, ведь нет никаких гарантий, что в каких-нибудь новостях не будет использовано какое-либо другое пространство имен (что вполне легально с точки зрения RDF и XML), и парсер пропустит все новости.
Если же XML-парсер понимает пространства имен, вы можете построить более изящное решение, которое сумеет разобрать новости и формате 0.91 и в формате 1.0.
\item Менее очевидный, но важный факт состоит в том, что в RSS 1.0 элементы item находятся вне элемента channel. В RSS 0.91 элементы item расположены внутри channel. В 0.90 они были снаружи. В 2.0 - они внутри.
\item В элементе channel есть один элемент items. Он нужен только для RDF-парсеров (задает порядок новостей).Можно его игнорировать и считать, что все новости идут в том порядке, в каком расположены элементы item.
\end {enumerate}
\par\noindent
\flushleft\parindent=1cm А как выглядит формат RSS 2.0? К счастью, для программ, понимающих форматы RSS 0.91 и 1.0, формат RSS 2.0 будет довольно простым:\par
	\center \includegraphics{RSStwo}   
           \center Рисунок 1.4 – RSS формат 2.0\par
\flushleft\parindent=1cm Как показывает данный пример, в RSS 2.0 тоже используются пространства имен, как и в RSS 1.0. Но это не RDF. Как и в RSS 0.91, нет пространства имен, принятого по умолчанию, а новости (в элементах item) размещены опять в элементе channel, что доказывает преимущество формата rss 2.0.\par
Изучив все форматы от rss 0.9 до rss 2.0, пришли к выводу, что в данной курсовой работе лучше всего рассматривать формат 2.0, так как в данный момент большинство сайтов используют именно данный формат для формирования своего RSS канала.\par
Теперь давайте взглянем, как та же самая информация выглядит в формате RSS 1.0:\par
	
           %--------------------------1.3 Конкуренция с форматом Atom----------------------------------
           \textbf{  1.3 Конкуренция с форматом Atom}
\flushleft\parindent=1cm В июне 2006 года появился конкурент RSS — формат Atom. Стандарт Atom 1.0 определяет 21 элемент канала новостей, RSS 2.0 — 30 элементов. Большинство элементов RSS 2.0 и Atom 1.0 не пересекаются и не соответствуют аналогам. Часть элементов редко используется на практике или их функциональность достигается другими путями. В середине 2005 года, когда стандарт Atom был только принят, он использовался лишь на нескольких десятках сайтов. В то же время RSS 2.0 был широко распространен на различных социальных сервисах. К началу 2006 года Atom 1.0 стал получать распространение и стал проблемой для программ-агрегатов, которые не могли разобрать формат, не похожий на RSS. Разработчики агрегатов принялись патчить свои программы на основе баг-репортов. К середине 2006 года, благодаря активному продвижению W3C и поддержке со стороны Google, IBM и еще ряда корпораций, Atom стал основным форматом для многих крупных онлайн-сервисов. При этом популярность RSS 2.0 не снизилась, так как контент-провайдеры зачастую стали представлять новости в обоих форматах, на выбор.\par
Канал RSS 2.0 может содержать только текстовую, или только гипертекстовую (HTML-escaped, теги экранируются в CDATA) информацию, без возможности указания, какое предоставление используется. Экранированные HTML (например, строка ATT будет представлена как ATamp;T) привносят дополнительные сложности разработчикам. Модель содержания заголовков (<title>) не определена. Заголовки с угловыми скобками или амперсандами будут интерпретированы значительной частью программ чтения независимо от представления. Модель содержания RSS 2.0 не допускает использования XML, что уменьшает возможности повторного использования содержимого.
В свою очередь, Atom имеет тщательно проработанный контейнер содержания, которое может быть следующих типов:
	\begin{enumerate}
		\item Простой текст, без разметки (по умолчанию).	
		\item Экранированный HTML.
		\item Правильно оформленная (well-formed) разметка XHTML.
		\item Некоторые другие XML словари.
		\item Двоичное содержание, закодированное в base64.
		\item Указатель на внешнее веб-содержимое, не включенное в канал.
	\end{enumerate}	
 \par\noindent 
 \parindent=1cm Atom не дает гарантии, что получатель будет иметь возможность сделать что-нибудь полезное с внешними данными или бинарным содержанием. Тем не менее, это ограждает получателей от неправильного определения типа содержимого на основе предположений.\par
 По результатам анализа и сравнения можно сделать следующие выводы:
 	\begin{enumerate}
		\item Стандарт Atom в силу своих возможностей является более универсальным средством синдикации и агрегирования информации. Этот формат следует использовать при разработке новых приложений, использующих каналы.	
		\item Спецификация RSS 2.0 остается широко распространенной и в ближайшее время останется таковой. Следовательно, разработчикам агрегаторов и роботов-сборщиков не стоит пока отказываться от этого формата синдикации и агрегирования.
	\end{enumerate}	
 \par\noindent 
           
\
	
            
         
  %-----2 ПОСТАНОВКА ЗАДАЧИ----
	\newpage
	\addcontentsline{toc}{section}{2 ПОСТАНОВКА ЗАДАЧИ}
	\section*{\normalsize\hspace{2ex}2 ПОСТАНОВКА ЗАДАЧИ}

В рамках данной курсовой работы необходимо разработать веб-сайт на тему формула 1 с RSS фидами. Данный сайт должен содержать:

\hspace {4ex}1.	информацию о командных и индивидуальных зачётах.\

\hspace{4ex}2.	информацию о последних гран-при.\

\hspace{4ex}3.	информацию о пилотах и командах формулы 1.\

\hspace{4ex}3.	ссылки на RSS каналы интересующих сайтов.\

Для разработки программы использовать среду язык программирования ASP.NET WEB SITE  и, соответственно база данных MsSQL. Использования среды разработки язык программирования c \#  позволяет быстро и качественно создавать логику сайтов, а использования  HTML и CCS позволяет сделать и развитый и качественный графический пользовательский интерфейс.
\




	%-----3 РАЗРАБОТКА ИГРОВОГО ПРИЛОЖЕНИЯ----
	\newpage
	\section*{\normalsize\hspace{3ex} 3 РАЗРАБОТКА ПРИЛОЖЕНИЯ}
	\addcontentsline{toc}{section}{3  РАЗРАБОТКА ПРИЛОЖЕНИЯ}
	\addcontentsline{toc}{subsection}{3.1 Структура сайта}
	\addcontentsline{toc}{subsection}{3.2 Master Page}
	\addcontentsline{toc}{subsection}{3.3 Главная страница}
	\addcontentsline{toc}{subsubsection}{3.3.1 Класc NewsPublishing}
	\addcontentsline{toc}{subsubsection}{3.3.2 Класс DriversStandings}
	\addcontentsline{toc}{subsubsection}{3.3.3 Класс TeamStandings}
	\addcontentsline{toc}{subsection}{3.4 Страница Standings}
	\addcontentsline{toc}{subsection}{3.5 Страница Drivers}
	\addcontentsline{toc}{subsection}{3.6 Страница Schedule}
	\addcontentsline{toc}{subsubsection}{3.6.1 Класс GrandPrixList}
	\addcontentsline{toc}{subsection}{3.7 Страница LatestRaceResult}
	\addcontentsline{toc}{subsubsection}{3.7.1 Класс LatestRaceInfo}
	
	
	%-----3.1 Структура сайта----
	
	\textbf{3.1 Структура сайта}
	
\flushleft\parindent=1cm Структура сайта будет состоять из следующих страниц: главная страница, страница с индивидуальным зачётом пилотов формулы 1, страница с пилотами данного сезона и их индивидуальной информации, а также страница с расписанием всех гран-при текущего сезона и страницы с результатом последней гонки. Главная страница будет включать в себя информацию о текущем индивидуальном зачёте лучших десяти пилотов и кубке конструкторов данного года, а также список с RSS ссылками на интересующие новости. Страница с индивидуальным зачётом будет содержать в себе информацию о индивидуальном зачёте каждого пилота. Страница с расписанием содержит в себе дату проведения каждого гран-при, а также время и ссылку на более подробную информацию о данной трассе. Последняя страница меню содержит в себе информацию о последнем гран-при: победитель, стартовая позиция, финишная позиция, лучший круг, время, статус, команда, количество очков, заработанных за гонку.\par
	
	%-----3.2 Master Page----
	
	\textbf{3.2 Master Page}
\flushleft\parindent=1cm В основе каждой страницы лежит мастер-страница. Мастер-страница - это средство ASP.NET, разработанное специально для стандартизации компоновки веб-страниц. Мастер-страница представляет собой шаблоны веб-страниц, которые могут определять фиксированное содержимое и объявлять часть веб-страницы, куда можно помещать нестандартное содержимое. При использовании одной и той же мастер-страницы во всем веб-сайте компоновка гарантированно будет одинаковой. Более того, если изменить определение мастер-страницы после ее применения, то все использующие ее веб-страницы автоматически воспримут это изменение.
	Данная мастер страница состоит из верхнего контитула, контента и нижнего контитула. Верхний контитул содержит в себе строку с меню в виде списка.\par

	
	%-----3.3 Главная страница----
	
	\textbf{3.3 Главная страница}
\flushleft\parindent=1cm При запуске сайта открывается главная страница Default.aspx. Данная страница (рис 3.3.1) отображает информацию об индивидуальном и командном зачётах а также генерирует список новостей в случае выбора RSS ссылки из составленного списка. Данный список содержится в БД. Его можно считывать, а также добавлять и удалять информацию из него.\par
	\includegraphics{MainPage}   
           \center Рисунок 3.3.1 – главная страница сайта\par
           
\flushleft\parindent=1cm В случае загрузки главной страницы, срабатывает метод PageLoad, в котором формируется список зачётов пилотов - standingsDrivers при вызове метода MakeDriversStandings() класса DriversStandings. Также в результате срабатывания метода PageLoad формируется список кубка конструкторов – standingsTeam при вызове метода MakeTeamStandings() класса TeamStandings. После чего создаём подключение к БД, в которой хранится название сайта и его RSS ссылка (рисунок 3.3.2).\par
	\includegraphics{DataBase}
           \center Рисунок 3.3.2 – структура БД \par
           
\flushleft\parindent=1cm После создания подключения, считываем всю информацию с БД параллельно записывая в список DropDownListRssUrls url сайта и во временный список listRssUrls RSS ссылка сайта. Взаимодействие с БД выполнялось с использованием Entity Framework.\par
Добавление новой ссылки в базу данных осуществляется в результате заполнения двух информационных полей и нажатия кнопки “Add”, после чего на сервере срабатывает событие ButtonUrlAddClick(object sender, EventArgs e). Код события представлен на рисунке 3.3.3. В случае если все данные введены верно, мы создаём подключение к базе данных, усанавливаем значение свойств экземпляра класса News, добавляем его в базу данных и обновляем её.\par
	\center \includegraphics{AddRssToDb}\par   
           \center Рисунок 3.3.2 – добавление ссылки на новость в БД \par
           \flushleft\parindent=1cm В случае выбора одной строки списка DropDownListRssUrls срабатывает метод DropDownListRssUrlsSelectedIndexChanged(object sender, EventArgs e). Суть его метода заключается в том, чтобы по выбраному имени сайта установить RSS ссылку, с которой потом считываются данные. Чтение новостей происходит в результате вызова метода ReadingNewsFromRssChanel() класса NewsPublishing.Рассмотрим работу этого класса подробнее.\par
	%-----3.3.1 Класc NewsPublishing----
	\textbf{3.3.1 Класc NewsPublishing}
\flushleft\parindent=1cm Основная идея класса NewsPublishing заключается в построении списка посредством чтения новостей определённого RSS канала. RSS формат имеет весьма простую структуру. Основными её тегами являются: Title – заголовок статьи, Description – описание статьи, PublishDate – дата публикации, Link – непосредственно ссылка на новость. Список уже прочитанных новостей представляет собой список экземпляров класса News (рис 2), которые содержат информацию о конкретной новости. Помимо этого, в классе хранится список новостей текущей ссылки, сама текущая ссылка на новостную ленту, а также список всех RSS ссылок, которые были считаны с базы данных. Рассмотрим методы данного класса:\par
   \par
public static IEnumerable<Feed> GetNewsColection();\par
Назначение: возвращает список новостей текущей ссылки – CurrentRssUrl.\par
   \par
public static void SetRssUrlFromDropListIndex(int index);\par
Назначение: метод устанавливает значение текущей ссылки в зависимости от выбранного индекса всплывающего списка.\par
   \par
public static void AddUrlToList(string url);\par
Назначение: метод добавляет ссылку на новстную ленту в список всех лент.\par
   \par
public static void SetListRssUrls(List<String> newListUrls);\par
Назначение: метод утанавливает список ссылок новостных лент.\par
   \par
public static void RemoveUrlFromList(int index);\par
Наначение: метод удаляет ссылку на новость из списка ссылок всех новостей по выбраному индексу.\par
   \par
public static void ReadingNewsFromRssChanel();\par
Назначение: данный метод посылает запрос по текущей ссылке – CurrentRssUrl и получает ответ в формате RSS. Считав информацию со всех тегов Item, получаем список новостей данного RSS канала. Код данного метода отображается на рис 5.\par
	\includegraphics{ReadingNewsRss}\par   
           \center Рисунок 3.3.2 – чтение новостей из RSS канала \par
\flushleft\parindent=1cm Серверный c код главной страницы использует следующие классы логику которых рассмотрим подробнее: DriversStandings, TeamStandings.\par
	%-----3.3.2 Класс DriversStandings----
	
	\textbf{3.3.2 Класс DriversStandings}
\flushleft\parindent=1cm Данный класс формирует список пилотов индивидуального зачёта - List<Driver> standingsDrivers, используя метод MakeDriversStandings().\par
  \par
public static void MakeDriversStandings();\par
Назначение: данный метод отправляет запрос по адресу - http://ergast.com/api/f1/2015/driverStandings и получает данные в формате xml (рисунок 2). В документе содержится информация о пилотах данного сезона ( рисунок 3).Прочитав всю необходимую информацию из документа, формируется список List<Driver> standingsDrivers. Другие основные методы:\par
 \par
public static string GetCurrentDriverNumber();\par
Назначение: Возвращает номер выбранного гонщика.\par
 \par
public static IEnumerable<Driver> GetDriverStandings();\par
Назначение: Возвращает список List<Driver> standingsDrivers, так как данный список объявлен с модификатором доступа private.\par
 \par
private static string GetImageCar(string str);\par
Назначение: Возвращает путь к картинке болида пилота относительно корневого элемента.\par
 \par
private static string GetImageDriver(string number);\par
Назначение: Возвращает путь к картинке пилота относительно корневого элемента.\par
	\includegraphics{DriverStandingsXMLresponse}   
           \center Рисунок 3.3.2 – пример Xml ответа, индивидуальный зачёт \par
           \includegraphics{classDriver}
           \center Рисунок 3.3.2 – структура класса Driver \par
	%-----3.3.3 Класс TeamStandings----
	\textbf{3.3.3 Класс TeamStandings}
\flushleft\parindent=1cm Данный класс формирует список конструкторов - List<Team> standingsTeam, используя метод MakeTeamStandings().\par
 \par
public static void MakeTeamStandings();\par
Назначение: данный метод отправляет запрос по адресу - http://ergast.com/api/f1/current/constructorStandings и получает данные в формате xml (рисунок 4). В документе содержится информация о командах данного сезона (рисунок 5).Прочитав всю необходимую информацию из документа, формируется список List<Driver> standingsDrivers. Основные методы:\par
 \par
public static List<Team> GetTeamStandings();\par
Назначение: возвращает список конструкторов List<Driver> standingsDrivers.\par	
	\includegraphics{TeamStandingsXMLresponse}   
           \center Рисунок 3.3.2 – пример Xml ответа, командный зачёт \par
           \includegraphics{classTeam}
           \center Рисунок 3.3.2 – структура класса Team\par
	%-----3.4 Страница Standings----
	\textbf{3.4 Страница Standings}
\flushleft\parindent=1cm Данная страница отображает информацию о всех пилотах текущего сезона, а именно: картинка пилота его номер, команду и соответственно имя.
Данная страница генерирует html разметку через компонент Repeater, datasource которого устанавливается на стороне сервера.\par
В случае клика на определённого пилота появляется новая страничка – PersonalDriver.aspx с личной информацией данного пилота: команда, количество подиумов, количество очков, национальность, болид, шлем пилота. Зная номер пилота, можно отобразить информацию о нём.\par
	%-----3.5 Страница Drivers----
	\textbf{3.5 Страница Drivers}
\flushleft\parindent=1cm Данная страница отображает информацию о всех пилотах текущего сезона, а именно: картинка пилота его номер, команду и соответственно имя.
Данная страница генерирует html разметку через компонент Repeater, datasource которого устанавливается на стороне сервера.\par
В случае клика на определённого пилота появляется новая страничка – PersonalDriver.aspx с личной информацией данного пилота: команда, количество подиумов, количество очков, национальность, болид, шлем пилота. Зная номер пилота, можно отобразить информацию о нём.\par
	%-----3.6 Страница Schedule----
	\textbf{3.6 Страница Schedule}
\flushleft\parindent=1cm Содержимое этой страницы отображает календарь гран-при формулы 1 текущего сезона. В состав содержимого входят: номер гран-при, дата проведения, время проведения, название гран-при и ссылка на информацию в википедии о гран-при. При загрузке страницы срабатывает серверный код, в котором считывается расписание гран-при с сервера и записывается в список класса GrandPrixList и в кэш. Алгоритм представлен на рисунке 7.\par
	\includegraphics{ScheduleSh}   
           \center Рисунок 3.3.2 – событие PageLoad страницы Schedule \par
	%-----3.6.1 Класс GrandPrixList----
	\textbf{3.6.1 Класс GrandPrixList}
\flushleft\parindent=1cm Данный класс формирует расписание сезона формулы 1. Он содержит List<GrandPrix>  с информацией о каждом гран-при, который формируется посредством чтения  Xml документа, полученного из удалённого сервера.
Структура формата Xml представлена на рис 8.\par
	\includegraphics{ScheduleResponseXml}   
           \center Рисунок 3.3.2 – пример Xml ответа, расписание гран-при.\par
\flushleft\parindent=1cm Основные методы класса:\par
 \par
public static void MakeGrandPrixList();\par
Назначение: считывает расписание гран-при с удалённого сервера в формате xml, записывает всю информацию в список.\par

public static List<GrandPrix> GetScheduleList()\par
Назначение: возвращает список гран-при.Структура класса GranPrix рисунок 9.\par

public static void SetScheduleList(List<GrandPrix> grandPrixList);\par
Назначение: устанавливает список гран-при.\par

	\includegraphics{classGrandPrix}   
           \center Рисунок 3.3.2 – структура класса GrandPrix\par

	%-----3.7 Страница LatestRaceResult----
	\textbf{3.7 Страница LatestRaceResult}
\flushleft\parindent=1cm Данная страница отображает последний гран-при, дату проведения, и результаты последнего гран-при: позиция, имя, стартовая позиция, определяемая по результатам квалификации, команда пилота, количество кругов, статус, быстрые круги трассы, время гонки, и количество очков. При загрузке страницы срабатывает серверный код, в котором считываются результаты последней гонки и записываются в список класса LatestRaceInfo и в кэш. Алгоритм представлен на рисунке 10\par
Рассмотрим структуру класса LatestRaceInfo подробнее.\par
	%-----3.7.1 Класс LatestRaceInfo----



%-----4 РУКОВОДСТВО ПО УСТАНОВКЕ И ИСПОЛЬЗОВАНИЮ ВЕБ САЙТА------
	\newpage
	\addcontentsline{toc}{section}{4 РУКОВОДСТВО ПО УСТАНОВКЕ И ИСПОЛЬЗОВАНИЮ ВЕБ-САЙТА}
           \section*{\normalsize\hspace{4ex}4 РУКОВОДСТВО ПО УСТАНОВКЕ И ИСПОЛЬЗОВАНИЮ ВЕБ-САЙТА}
%-----4.1 Развёртывание веб-сайта------
           \addcontentsline{toc}{section}{  4.1 Развёртывание веб-сайта}
 	\section*{\normalsize\hspace{4ex}4.1 Развёртывание веб-сайта}
\flushleft\parindent=1cm Прежде чем развертывать веб-сайт, нужно подготовить IIS. Главное решение, которое нужно приять, касается места размещения содержимого и его влияния на конечный URL-адрес. Начнем с очевидного подхода - предположим, что необходимо, чтобы URL-адрес для содержимого данного примера был следующим: http://имя сервера:80/Website/F1\par
IIS нужно подготовить так, чтобы было куда скопировать наш файл. В IIS Manager выберите элемент Default Web Site. Как следует из его имени, это сайт по умолчанию на сервере. Щелкните на нем правой кнопкой мыши и в контекстном меню выберите пункт Explore (Проводник), чтобы отрыть окно проводника Windows для заданного по умолчанию каталога IIS, которым является inetpub wwwroot на системном томе (как правило, C:).\par
Создайте каталог Website, а в нем - каталог F1(чтобы обеспечить существование пути inetpub wwwroot-Website-F1). Закройте окно проводника, чтобы вернуться в IIS Manager. Щелкните правой кнопкой на записи Default Web Site и в контекстном меню выберите пункт Refresh (Обновить), чтобы увидеть новый каталог. Переместите файлы веб-сайта на сервер любым подходящим способом - посредством общего сетевого диска, съемного диска USB и т.п. - и скопируйте файлы сайта в каталог F1, созданный на сервере.\par
IIS понадобится также указать, что развернутый сайт является приложением. Это не обязательно, но при развертывании приложений ASP.NET почти всегда будет желательным - активизируется состояние сеанса и другие функциональные средства ASP.NET. Щелкните правой кнопкой мыши на папке F1 в области Connections (Подключения) и в контекстном меню выберите пункт Convert to Application (Преобразовать в приложение). Откроется диалоговое окно Add Application (Добавление приложения). Используемый пул приложений можно изменить, щелкнув на кнопке Select (Выбрать). Настроить учетную запись пользователя, которую IIS будет применять для доступа к содержимому сайта, можно с помощью кнопок Connect as... (Подкл. как...) и Test Settings... (Тест настроек...). Пока что просто щелкните на кнопке ОК. Возможно, придется выбрать пункт Refresh (Обновить) в меню View (Вид) (или, как это часто имеет место, закрыть и снова открыть IIS Manager), но теперь значок записи FileCopy в древовидном представлении должен измениться.\par
%-----5.2 Работа с приложением------
	\addcontentsline{toc}{section}{  4.2 Работа с приложением}
 	\section*{\normalsize\hspace{4ex}4.2 Работа с приложением}
\flushleft\parindent=1cm В случае входа на сайт пользователю предоставляется информация о последних гонках, расписании, индивидуальном зачёте пилотов, кубке конструкторов, а также информация о пилотах текущего сезона формулы 1.\par
Кроме того, пользователь может следить за новостями интересующего сайта, просто добавив ссылку на RSS канал в список фидов (рисунок 11). \par
	\includegraphics{OldList}   
           \center Рисунок 3.3.2 – добавление нового фида в список\par
Выбрав нужный фид из переченя, пользователю предоставляются последние новости сайта, без его посещения(рисунок 12).\par
	\includegraphics{News}   
           \center Рисунок 3.3.2 –  новости с сайта f1-world.ru\par
Также, пользователь может удалять сайт из списка.\par
	%-------ЗАКЛЮЧЕНИЕ-------
	\newpage
	\addcontentsline{toc}{section}{ЗАКЛЮЧЕНИЕ}
	\section*{\center\normalsize ЗАКЛЮЧЕНИЕ \endcenter}
	\hspace{4ex}В результате выполнения данной курсовой работы был создан сайт на тему формулы 1 с чтением RSS фидов, который может быть использован в личных целях для отслеживания последних новостей любимых сайтов, не посещая их, что экономит не мало времени на посещение этого сайта и трафика для открытия всей страницы.\
	\hspace{4ex}В ходе выполнения данной курсовой работы были получены навыки работы с ASP.NET WEB SITE. Детально изучены язык гипертекстовой разметки HTML и таблица каскадных стилей CSS.Получены ознокомительные навыки с базой данных MsSql и средством для работы с базой данных – Entity Framework.\

\hspace{4ex}Возможности сайта:\

\hspace{4ex}– Показ текстовой и графической информации.

\hspace{4ex}– Использование кэша.

\hspace{4ex}– Считывание данных с удалённого сервера в формате Xml.

\hspace{4ex}– Работа с базой данных.

\hspace{4ex}– Отображение новостей интересующего сайта.

\hspace{4ex}– Добавление и удаление интересующих фидов.

\hspace{4ex}– Работа с классами.

		\hspace{4ex} При дальнейшем развитие проекта будут добавлены следующие функции:

\hspace{4ex}– Расширенная работа с JavaScript.

\hspace{4ex}– Оптимизация работы с базой данных.\

\hspace{4ex}–  Улучшенный дизайн.

\hspace{4ex}–  Возможность просмотра результатов гонок предыдущих сезонов, а также статистики всех пилотов истории формулы 1.





	%----СПИСОК ИСПОЛЬЗОВАННОЙ ЛИТЕРАТУРЫ-------
	\newpage
	\addcontentsline{toc}{section}{СПИСОК ИСПОЛЬЗОВАННОЙ ЛИТЕРАТУРЫ}
	\section*{\center\normalsize СПИСОК ИСПОЛЬЗОВАННОЙ ЛИТЕРАТУРЫ \endcenter}

\hspace{4ex}[1] С.Г.Баричев, В.В.Гончаров, Р.Е.Серов. Основы PHP – Москва, 2001. – 56 с.\
 
\hspace{4ex}[2] Торонский В. Коммерческие системы шифрования: основные алгоритмы и их реализация. Часть 1.  Монитор. - 2002. - N 6-7. - c. 14 - 19.\

\hspace{4ex}[3] Н. Робин , Создаем динамические Веб-страницы, М.: научное изд-во ТВП, 2001. – 100 с.\

\hspace{4ex}[4] Куликов C.C, Курс лекций.\

\hspace{4ex}[5] ГОСТ 19.701–90. Единая система программной документации. Схемы алгоритмов, программ, данных и систем. Условные обозначения и правила выполнения. – Введ. 1992–01–01. – М. : Изд-во стандартов, 1991. – 450 с.\

\hspace{4ex}[6] ГЛинн Бейли, Майкл Моррисон. Изучаем PHP и MySQL Эксмо
2004. – 410 с.\

\hspace{4ex}[7] Николай Прохоренок. HTML, JavaScript, PHP и MySQL. Джентльменский набор Web-мастера. БХВ-Петербург. 2010 - 214с.\






	%----ПРИЛОЖЕНИЕ А (обязательное) Исходный код программы----
	\begin{landscape}
	\newpage
	\addcontentsline{toc}{section}{ПРИЛОЖЕНИЕ А}
	\section*{\center\normalsize ПРИЛОЖЕНИЕ А\\(обязательное)\\Исходный код программы \endcenter}
	<html>HTML-код</html>
	\end{landscape}
	
	
\end{document}          
