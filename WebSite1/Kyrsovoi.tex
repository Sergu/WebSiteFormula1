\documentclass[14pt,a4paper]{extreport}
\usepackage[top=2cm, left=3cm, bottom=2cm, right=1cm]{geometry}
\usepackage[utf8x]{inputenc} % Включаем поддержку UTF8
\usepackage[russian]{babel} % Пакет поддержки русского языка
\usepackage{lscape}
\usepackage{fancyhdr}
\usepackage{textcase}
\graphicspath{{pictures/}}
\DeclareGraphicsExtensions{.pdf,.png,.jpg}
\usepackage{caption}
\usepackage{graphicx}
\usepackage{caption}
\usepackage{refstyle}
\usepackage{color} 
\usepackage{listings} 

\lstset{
language=Matlab,
extendedchars=\true,
inputencoding=utf8x,
commentstyle=\itshape,
stringstyle=\bf,
belowcaptionskip=5pt }
\lstset{
literate={а}{{\selectfont\char224}}1
{б}{{\selectfont\char225}}1
{в}{{\selectfont\char226}}1
{г}{{\selectfont\char227}}1
{д}{{\selectfont\char228}}1
{е}{{\selectfont\char229}}1
{ё}{{\"e}}1
{ж}{{\selectfont\char230}}1
{з}{{\selectfont\char231}}1
{и}{{\selectfont\char232}}1
{й}{{\selectfont\char233}}1
{к}{{\selectfont\char234}}1
{л}{{\selectfont\char235}}1
{м}{{\selectfont\char236}}1
{н}{{\selectfont\char237}}1
{о}{{\selectfont\char238}}1
{п}{{\selectfont\char239}}1
{р}{{\selectfont\char240}}1
{с}{{\selectfont\char241}}1
{т}{{\selectfont\char242}}1
{у}{{\selectfont\char243}}1
{ф}{{\selectfont\char244}}1
{х}{{\selectfont\char245}}1
{ц}{{\selectfont\char246}}1
{ч}{{\selectfont\char247}}1
{ш}{{\selectfont\char248}}1
{щ}{{\selectfont\char249}}1
{ъ}{{\selectfont\char250}}1
{ы}{{\selectfont\char251}}1
{ь}{{\selectfont\char252}}1
{э}{{\selectfont\char253}}1
{ю}{{\selectfont\char254}}1
{я}{{\selectfont\char255}}1
{А}{{\selectfont\char192}}1
{Б}{{\selectfont\char193}}1
{В}{{\selectfont\char194}}1
{Г}{{\selectfont\char195}}1
{Д}{{\selectfont\char196}}1
{Е}{{\selectfont\char197}}1
{Ё}{{\"E}}1
{Ж}{{\selectfont\char198}}1
{З}{{\selectfont\char199}}1
{И}{{\selectfont\char200}}1
{Й}{{\selectfont\char201}}1
{К}{{\selectfont\char202}}1
{Л}{{\selectfont\char203}}1
{М}{{\selectfont\char204}}1
{Н}{{\selectfont\char205}}1
{О}{{\selectfont\char206}}1
{П}{{\selectfont\char207}}1
{Р}{{\selectfont\char208}}1
{С}{{\selectfont\char209}}1
{Т}{{\selectfont\char210}}1
{У}{{\selectfont\char211}}1
{Ф}{{\selectfont\char212}}1
{Х}{{\selectfont\char213}}1
{Ц}{{\selectfont\char214}}1
{Ч}{{\selectfont\char215}}1
{Ш}{{\selectfont\char216}}1
{Щ}{{\selectfont\char217}}1
{Ъ}{{\selectfont\char218}}1
{Ы}{{\selectfont\char219}}1
{Ь}{{\selectfont\char220}}1
{Э}{{\selectfont\char221}}1
{Ю}{{\selectfont\char222}}1
{Я}{{\selectfont\char223}}1
}
\usepackage{caption}
\DeclareCaptionFont{white}{\color{white}} %% это сделает текст заголовка белым
%% код ниже нарисует серую рамочку вокруг заголовка кода.
\DeclareCaptionFormat{listing}{\colorbox{gray}{\parbox{\textwidth}{#1#2#3}}}
\captionsetup[lstlisting]{format=listing,labelfont=white,textfont=white}

\title{}
\author{}

\begin{document}

%----------ТИТУЛЬНЫЙ-ЛИСТ--------------
	\center
	Министерство образования Республики Беларусь\\
	Учреждение образования «Белорусский государственный университет информатики и радиоэлектроники»
	\vspace*{2cm}
	\endcenter
	\raggedright
	Факультет компьютерных систем и сетей\\
	\medskip
	Кафедра программного обеспечения информационных технологий\\
	\medskip
	Дисциплина:  Компьютерные системы и сети (КСиС)
	\vspace*{2cm}
	\center
	ПОЯСНИТЕЛЬНАЯ ЗАПИСКА\\
	к курсовому проекту\\
	на тему\\
	\medskip
	Сайт формулы 1 c чтением RSS фидов\\
	\medskip
	БГУИР КП  1-40 01 0 22 ПЗ
	\vspace*{4cm}
	\endcenter
	\raggedright
	\hspace*{7.94cm} Студент: гр. 351002 Сергушов Н.О.\\
	\bigskip
	\hspace*{7.94cm}Руководитель: асс. Третьяков Ф.И.\\
	\center
	\vspace*{7cm}
	Минск 2015
	\pagestyle{empty}
%-------ЛИСТ-ЗАДАНИЯ--------------
	\newpage
	\center
	Учреждение образования\\
	\medskip
	«Белорусский государственный университет информатики и радиоэлектроники»\\
	\medskip
	Факультет компьютерных систем и сетей\\
	\medskip
	\endcenter
	\raggedright
	\hspace*{9.53cm}УТВЕРЖДАЮ\\
	\hspace*{9.53cm}Заведующий кафедрой ПОИТ\\
	\hspace*{9.53cm}\underline{\hspace{6cm}} \\
	\hspace*{11cm}\small (подпись) \normalsize\\
	\hspace*{9.53cm}\underline{\hspace{5cm}}2015 г.\\
	\medskip
	\center
	ЗАДАНИЕ\\
	по курсовому проектированию\\
	\medskip
	\endcenter
	\raggedright
	Студенту \underline{ Сергушову Никите Олеговичу }\\
	\begin{enumerate}
	\item Тема работы \underline{Сайт формулы 1 с чтением RSS фидов}\\ 
	\item Срок сдачи студентом законченной работы \underline{09.06.2015}
	\item Исходные данные к работе \underline{Среда разработки Visual Studio 2013. }
	\item Содержание расчётно-пояснительной записки (перечень вопросов, которые подлежат разработке)\\
	\underline{\hspace*{16cm}}\hspace*{-16cm}Введение.\par
	\underline{\hspace*{16cm}}\hspace*{-16cm}1.Аналитический обзор литературы\par
	\underline{\hspace*{16cm}}\hspace*{-16cm}2.Постановка задачи\par
	\underline{\hspace*{16cm}}\hspace*{-16cm}3.Разработка программного средства\par
	\underline{\hspace*{16cm}}\hspace*{-16cm}4.Руководство пользователя\par
	\underline{\hspace*{16cm}}\hspace*{-16cm}Заключение\par
	\underline{\hspace*{16cm}}\hspace*{-16cm}Приложения\par
	\item Перечень графического материала (с точным обозначением обязательных чертежей и графиков)\\
	\underline{1. Схема алгоритма}
	\item Консультант по курсовой работе\\
	\underline{Третьяков Ф.И.}  
	\item Дата выдачи задания \underline{20.02.2015}
	\item Календарный график работы над проектом на весь период проектирования (с обозначением сроков выполнения и процентом от общего объёма работы):\\
	\underline{\hspace*{16cm}}\hspace*{-16cm}раздел 1 к DD.MM.YYYY – 15 \% готовности работы;\\  
	\underline{\hspace*{16cm}}\hspace*{-16cm}разделы 2, 3 к DD.MM.YYYY – 30 \% готовности работы;\\ 
	\underline{\hspace*{16cm}}\hspace*{-16cm}раздел 4 к DD.MM.YYYY – 60 \% готовности работы;\\
	\underline{\hspace*{16cm}}\hspace*{-16cm}раздел 5, 6 к DD.MM.YYYY  –  90 \% готовности работы;\\
	\underline{\hspace*{16cm}}\hspace*{-16cm}оформление пояснительной записки и графического материала к\\
	\underline{\hspace*{16cm}}\hspace*{-16cm}DD.MM.YYYY – 100 \% готовности работы.\\
	\underline{\hspace*{16cm}}\hspace*{-16cm}Защита курсового проекта с 5 по 9 июня 2015 г.\\
	\end{enumerate}
	\hspace*{7cm}РУКОВОДИТЕЛЬ\underline{\hspace*{6cm}}\hspace*{-3.9cm}Третьяков Ф.И.\\
	\hspace*{11.5cm}\small (подпись) \normalsize\\
	\bigskip
	Задание принял к исполнению \underline{\hspace*{10.5cm}}\hspace*{-8cm}Сергушов Н.О. DD.MM.YYYYг.\\
	\hspace*{7cm}\small (дата и подпись студента) \normalsize\\
	%-------СОДЕРЖАНИЕ--------------
	\newpage
	\pagestyle{plain}
	%\renewcommand{\headrulewidth}{0px}
	%\fancypagestyle{plain}{\cfoot{}\rfoot{\thepage}}
	
	\renewcommand\contentsname{\center\normalsize \textbf{СОДЕРЖАНИЕ} \endcenter}
	\tableofcontents
	\endcenter
	



	%-----ВВЕДЕНИЕ-----
	\newpage
	\addcontentsline{toc}{section}{ВВЕДЕНИЕ}
	\section*{\center\normalsize ВВЕДЕНИЕ \endcenter}
	\flushleft\hspace{4ex} Чемпионат мира «Формулы-1» проводится каждый год и состоит из отдельных этапов (имеющих статус Гран-при). В конце года выявляется победитель чемпионата. В Формуле-1 соревнуются как отдельные пилоты, так и команды. Пилоты соревнуются за титул чемпиона мира, а команды — за Кубок конструкторов. Команды, участвующие в гонках Формулы-1, используют на Гран-при болиды (гоночные автомобили) собственного производства. Таким образом, задачей команды является не только нанять быстрого и опытного пилота и обеспечить грамотную настройку и обслуживание машины, но и вообще «с нуля» спроектировать и сконструировать болид. Болиды участников чемпионата должны соответствовать техническому регламенту «Формулы-1» и пройти тест на ударопрочность. От каждой команды в каждом Гран-при должны выступать два гонщика, при этом раскраска машин должна быть одинаковой (за исключением номеров). В случае, если команда не выйдет на старт гонки или выставит только один болид, это может караться штрафом. Гран-при проводится с пятницы по воскресенье (за исключением Гран-При Монако, где свободные заезды производятся в четверг) и состоит из свободных заездов, квалификации и гонки. Сезон состоит из различного количества Гран-при: от 7 в 1950 до 20 в 2012 и обычно проводится с марта по ноябрь. За первые 10 мест на финише гонки пилоты и команды получают очки по системе 25-18-15-12-10-8-6-4-2-1. Трое первых гонщиков поднимаются на подиум. Очки, набранные пилотами в гонке, прибавляются к их счёту в борьбе за титул чемпиона мира, а также к счёту их команд в борьбе за Кубок конструкторов. На Гран-при от одной команды могут выступать 2 пилота в квалификации и гонке. Кроме того, в свободных заездах в пятницу команда может использовать третьего пилота, но принимать участие могут только 2 машины.\par
	\center \includegraphics{F1GrandPrix}\par
	\center Рисунок 1.1 Страны, где проводятся Гран-при Формулы-1\par
	
	\flushleft\hspace{4ex} В случае болезни или иных уважительных причин третий пилот может заменить на квалификации и гонке одного из основных пилотов команды, но 3-й пилот может быть допущен к гонке только если он участвовал в квалификационных заездах. Очки, набранные запасным пилотом в гонке, будут начислены на его личный счёт в чемпионате мира; в борьбе за Кубок конструкторов эти очки будут прибавлены к счёту команды, как если бы за неё выступал основной пилот. В течение сезона за одну команду могут выступать, набирая для себя и для неё очки, до 4 пилотов.\par
	\hspace{4ex}За многолетнюю историю Формулы-1 формат квалификации изменялся несколько раз. Квалификация проводится в субботу и состоит из сессий. Первая сессия длится 18 минут. Вторая сессия короче — 15 минут. Третья — самая короткая — 12 минут. Во время каждой сессии пилот может проехать столько кругов, сколько захочет и успеет. В зачёт пойдёт круг с наименьшим временем. В первой сессии участвуют все пилоты. Они проезжают произвольное число кругов. Для каждого пилота выбирается лучшее время прохождения круга, и последние 6 пилотов выбывают из борьбы за стартовую позицию, занимая места с 17-го по 22-е. Во второй сессии принимают участие оставшиеся 16 пилотов, снова проезжающих круги в произвольном режиме. Опять выбывают 6 пилотов, показавших худшее время по итогам данной сессии. Они занимают места с 11-го по 16-е. В третьей сессии принимают участие 10 пилотов, показавших лучшее время прохождения круга во второй сессии. Они соревнуются между собой на тех же условиях, что и ранее. Снова сравнивают лучшее время прохождения круга, определяя пилотов, занимающих первые 10 мест на старте гонки. Эта процедура применима в случае участия 22-х машин, если участвуют 20 машин, тогда после первой и второй сессии исключаются по пять участников.\par
	\hspace{4ex}После окончания квалификации болиды первой десятки отправляются в закрытый парк, то есть до начала гонки нельзя делать какие-либо изменения, то есть: менять состав резины, вносить коррективы в настройки шасси, менять настройки двигателя и т. п. Разрешаются лишь незначительные изменения настроек, и под чутким наблюдением инспекторов FIA. Первая десятка должна стартовать на тех же шинах, которые были установлены на болиде в тот момент второй сессии, когда было показано лучшее время. В исключительном случае комплект шин может быть заменен на аналогичный, если шины были повреждены, по согласованию с техническим делегатом ФИА. Остальные пилоты, занимающие места на стартовой решётке с 11-го и ниже, имеют право выбрать тот тип резины, который захотят. Если квалификация прошла на сухой трассе, а перед началом гонки она объявлена дождевой, то на всех машинах должны быть установлены дождевые шины и наоборот. В случае наложения стюардами на пилотов или команду штрафов в виде лишения нескольких мест на старте, эти штрафы применяются непосредственно перед стартом гонки, то есть квалификация проходит без их учёта. Чаще всего штрафы получают за блокировку гонщика на быстром круге и за замену двигателя или коробки передач.\par
	\flushleft\hspace{4ex} Rss технология весьма простая и эффективная. Чтобы начать ей пользоваться не нужно никаких усилий. Прежде всего, нужно осознать, как работает эта система. Это позволит понять и оценить всю пользу и ценность новостных лент.
Rss – это xml-формат данных, который позволяет следить за обновлениями сайтов в интернете. Несколько лет назад любой серьезный сайт предоставлял возможность подписки на свой RSS-канал. Сейчас наблюдается тенденция «устаревания», на сайтах перестали делать rss-ленты и привычную оранжевую иконку можно обнаружить все реже.   Но сам RSS от этого не стал хуже.\par

\hspace{4ex}На каждом сайте периодически публикуется какой-то контент — статьи, новости, комментарии. Подписка на RSS-ленту новостей сайта позволяет узнавать об обновлениях на этом сайте максимально оперативно, без необходимости посещать и просматривать сам сайт. Это сродни просмотру заголовков утренней газеты. Зачем идти на сайт и просматривать его в поисках новостей, если можно просматривать анонсы этих статей у себя на компьютере, и в случае находки интересного материала пройти по ссылке на свежую статью, заголовок которой обязательно будет в RSS-ленте.  Мало того, что это сберегает кучу времени, так еще и уменьшает трафик, ибо вы идете на сайт только если вам действительно что-то интересно, наверняка у вас есть любимые сайты, которые вы периодически посещаете, читаете статьи и новости, следите за обновлениями.

\hspace{4ex}Ленты новостей обрабатываются специальными программами — RSS-агрегатами.  Агрегаты бывают двух видов — web-сервисы и программы на вашем компьютере. Какой из них выбрать — это дело вкуса. Онлайн-агрегаты по функциональности чем-то похожи на почтовые сервисы — вы создаете аккаунт в Сети на одном из сервисов, и все ваши ленты будут приходить в ящик на вашем аккаунте. Этот способ хорош тем, что вы всегда будете иметь доступ к вашим лентам, независимо от того где вы находитесь. Но у них есть недостаток, что все же нужно идти и проверять ящик. То есть, чтобы узнать о новостях или подписаться на новую ленту вам открывать в браузере страницу сервиса-агрегата.

\hspace{4ex}Поэтому есть еще более удобные — локальные RSS-агрегаты, которые могут быть встроенными в браузер, почтовик или быть отдельным приложением.  Если вы знаете что такое почтовая программа, то вы поймете преимущество новостных лент. Локальный агрегат позволяет следить за новостями гораздо эффективнее — вы получаете обновления на сайтах практически моментально. Локальные RSS-агрегаты сами «бегают» на сайт и, если там появилось что-то новенькое «приносят» вам заголовок или анонс статьи, а иногда и всю статью — это зависит от того какую ленту предоставляет сайт.

\hspace{4ex}Цель данной курсовой работы заключается в создании веб-сайта F1, который будет отображать новости интересующих RSS-каналов. 

\hspace{4ex}Данная пояснительная записка состоит из ряда разделов, которые охватывают как теоретические, так и практические аспекты разработки программного средства. 
\




	%-----1 АНАЛИЗ ЛИТЕРАТУРНЫХ ИСТОЧНИКОВ----
	
	\newpage
	\addcontentsline{toc}{section}{1 АНАЛИЗ ЛИТЕРАТУРНЫХ ИСТОЧНИКОВ}

	\item\section*{\normalsize\hspace{4ex}АНАЛИЗ ЛИТЕРАТУРНЫХ ИСТОЧНИКОВ}
	
	%--------------------------1.1 Обзор аналогов----------------------------------
	\addcontentsline{toc}{subsection}{1.1 Обзор аналогов}
	\flushleft\hspace{4ex}\textbf{1.1  Обзор аналогов}
	
\flushleft\hspace {4ex}На данный момент из популярных браузеров возможность обработки rss ленты новостей реализована только в Mozilla.

\hspace{4ex}Firefox умеет обрабатывать rss-ленту в xml-формате. В случае использования браузера Firefox, при попытке подписаться на RSS, откроется страничка, где можно выбрать онлайн-сервис новостных лент. Сам Mozilla Firefox поддерживает подписку на RSS только в виде закладок на новости.

\hspace{4ex}Браузер Google Сhrome также не умеет работать c RSS-каналами, но такой функционал можно в него добавить в виде расширений.

\hspace{4ex}Онлайн-сервисы для подписки на RSS-каналы абсолютно бесплатны, и чем-то похожи на почтовые. Самые популярный это — Яндекс. Лента.  Для того чтобы воспользоваться сервисом нужно иметь Yandex аккаунт. Если вы уже пользуетесь какими-либо сервисами Яндекса, то нужно лишь войти под своим логином и подписаться на ленту. После чего можно читать новостную ленту через интерфейс Яндекс-почты.

	%--------------------------1.2 RSS формат----------------------------------
	\addcontentsline{toc}{subsection}{1.2 RSS формат}
	\flushleft\hspace{4ex}\textbf{1.2  Rss формат}\par
           \flushleft\hspace{4ex}На данный момент существует 7 различных форматов RSS.Их описание приведено на рисунке 1.2.
           \center \includegraphics{RSS_ALL}\par
	\center Рисунок 1.2 – RSS форматы\par
\flushleft\hspace {4ex}В данной курсовой работе лучше всего рассматривать формат 2.0, так как в данный момент большинство сайтов используют именно данный формат для формирования своего RSS канала.    


	%--------------------------1.3 Веб-технология ASP.NET----------------------------------
	\addcontentsline{toc}{subsection}{1.3 Веб-технология ASP.NET}
	\flushleft\hspace{4ex}\textbf{1.3 Веб-технология ASP.NET}

	\flushleft\hspace{4ex}ASP.NET  - технология создания веб-приложений и веб-сервисов от компании Майкрософт. Она является составной частью платформы Microsoft .NET и развитием более старой технологии Microsoft ASP. На данный момент последней версией этой технологии является ASP.NET 4.5.\par
 	\hspace{4ex}Хотя ASP.NET берёт своё название от старой технологии Microsoft ASP, она значительно от неё отличается. Microsoft полностью перестроила ASP.NET, основываясь на Common Language Runtime (CLR), которая является основой всех приложений Microsoft .NET. Разработчики могут писать код для ASP.NET, используя практически любые языки программирования, входящие в комплект .NET Framework (C\#, Visual Basic.NET и JScript .NET). ASP.NET имеет преимущество в скорости по сравнению со скриптовыми технологиями, так как при первом обращении код компилируется и помещается в специальный кэш, и впоследствии только исполняется, не требуя затрат времени на парсинг и оптимизацию.\par
	\hspace{4ex}Преимущества ASP.NET:\par
\begin{itemize}
	\itemКомпилируемый код выполняется быстрее, большинство ошибок отлавливается ещё на стадии разработки;
	\itemЗначительно улучшенная обработка ошибок во время выполнения запущеной готовой программы, с использованием блоков try..catch;
	\itemПользовательские элементы управления (controls) позволяют
выделять часто используемые шаблоны, такие как меню сайта;
	\itemИспользование метафор, уже применяющихся в Windows-приложениях, например, таких как элементы управления и события;
	\itemРасширяемый набор элементов управления и библиотек классов позволяет быстрее разрабатывать приложения;
	\item ASP.NET опирается на многоязыковые возможности .NET, что позволяет писать код страниц на VB.NET, Delphi.NET, Visual C\#, J\# и т. д;
	\itemВозможность кэширования всей страницы или её части для увеличения производительности;
	\itemВозможность кэширования данных, используемых на странице;
	\itemВозможность разделения визуальной части и бизнес-логики по разным файлам («code behind»);
	\itemРасширяемая модель обработки запросов;
	\itemРасширенная событийная модель;
	\itemРасширяемая модель серверных элементов управления;
	\itemНаличие master-страниц для задания шаблонов оформления страниц; 
\end{itemize}      
         
  %-----2 ПОСТАНОВКА ЗАДАЧИ----
	\newpage
	\addcontentsline{toc}{section}{2 ПОСТАНОВКА ЗАДАЧИ}
	\section*{\normalsize\hspace{4ex}2 ПОСТАНОВКА ЗАДАЧИ}

\flushleft\hspace{4ex}В рамках данной курсовой работы необходимо разработать веб-сайт на тему формула 1 с сылками на RSS каналы, который может быть использован в личных целях для отслеживания последних новостей интересующих сайтов. Данный сайт должен содержать:

\hspace {4ex}1.	информацию о командных и индивидуальных зачётах.\

\hspace{4ex}2.	информацию о последних гран-при.\

\hspace{4ex}3.	информацию о пилотах и командах формулы 1.\

\hspace{4ex}3.	чтение новостей с RSS каналов интересующих сайтов.\




	%-----3 РАЗРАБОТКА  ПРИЛОЖЕНИЯ----
	\newpage
	\section*{\normalsize\hspace{4ex}3 РАЗРАБОТКА ПРИЛОЖЕНИЯ}
	\addcontentsline{toc}{section}{3  РАЗРАБОТКА ПРИЛОЖЕНИЯ}
	
	
	%-----3.1 Структура сайта----
	\addcontentsline{toc}{subsection}{3.1 Структура сайта}
	\flushleft\hspace{4ex}\textbf{3.1 Структура сайта}
	
\flushleft\hspace{4ex}Структура сайта будет состоять из следующих страниц: главная страница, страница с индивидуальным зачётом пилотов формулы 1, страница с пилотами данного сезона и их индивидуальной информации, а также страница с расписанием всех гран-при текущего сезона и страницы с результатом последней гонки. Главная страница будет включать в себя информацию о текущем индивидуальном зачёте лучших десяти пилотов и кубке конструкторов данного года, а также список с RSS ссылками на интересующие новости. Страница с индивидуальным зачётом будет содержать в себе информацию о индивидуальном зачёте каждого пилота. Страница с расписанием содержит в себе дату проведения каждого гран-при, а также время и ссылку на более подробную информацию о данной трассе. Последняя страница меню содержит в себе информацию о последнем гран-при: победитель, стартовая позиция, финишная позиция, лучший круг, время, статус, команда, количество очков, заработанных за гонку.\par
	
	%-----3.2 Master Page----
	\addcontentsline{toc}{subsection}{3.2 Master Page}
	\flushleft\hspace{4ex}\textbf{3.2 Master Page}
\flushleft\hspace{4ex}В основе каждой страницы лежит мастер-страница. Мастер-страница - это средство ASP.NET, разработанное специально для стандартизации компоновки веб-страниц. Мастер-страница представляет собой шаблоны веб-страниц, которые могут определять фиксированное содержимое и объявлять часть веб-страницы, куда можно помещать нестандартное содержимое. При использовании одной и той же мастер-страницы во всем веб-сайте компоновка гарантированно будет одинаковой. Более того, если изменить определение мастер-страницы после ее применения, то все использующие ее веб-страницы автоматически воспримут это изменение.
	Данная мастер страница состоит из верхнего контитула, контента и нижнего контитула. Верхний контитул содержит в себе строку с меню в виде списка.\par

	
	%-----3.3 Главная страница----
	\addcontentsline{toc}{subsection}{3.3 Главная страница}
	\flushleft\hspace{4ex}\textbf{3.3 Главная страница}
\flushleft\hspace{4ex}При запуске сайта открывается главная страница Default.aspx. Данная страница (рис 3.3.1) отображает информацию об индивидуальном и командном зачётах а также генерирует список новостей в случае выбора RSS ссылки из составленного списка. Данный список содержится в БД. Его можно считывать, а также добавлять и удалять информацию из него.\par
	\includegraphics[scale=0.5]{MainPage}   
           \center Рисунок 3.3.1 – главная страница сайта\par
           
\flushleft\hspace{4ex} В случае загрузки главной страницы, срабатывает метод PageLoad, в котором формируется список зачётов пилотов - standingsDrivers при вызове метода MakeDriversStandings() класса DriversStandings. Также в результате срабатывания метода PageLoad формируется список кубка конструкторов – standingsTeam при вызове метода MakeTeamStandings() класса TeamStandings. После чего создаём подключение к БД, в которой хранится название сайта и его RSS ссылка (рисунок 3.3.2).\par
	\center\includegraphics{DataBase}\par
           \center Рисунок 3.3.2 – структура БД \par
           
\flushleft\hspace{4ex}После создания подключения, считываем всю информацию с БД параллельно записывая в список DropDownListRssUrls url сайта и во временный список listRssUrls RSS ссылка сайта. Взаимодействие с БД выполнялось с использованием Entity Framework.\par
\hspace{4ex}Добавление новой ссылки в базу данных осуществляется в результате заполнения двух информационных полей и нажатия кнопки “Add”, после чего на сервере срабатывает событие ButtonUrlAddClick(object sender, EventArgs e). В случае если все данные введены верно, мы создаём подключение к базе данных, усанавливаем значение свойств экземпляра класса News, добавляем его в базу данных и обновляем её.\par
\hspace{4ex}В случае выбора одной строки списка DropDownListRssUrls срабатывает метод DropDownListRssUrlsSelectedIndexChanged(object sender, EventArgs e). Суть его метода заключается в том, чтобы по выбраному имени сайта установить RSS ссылку, с которой потом считываются данные. Чтение новостей происходит в результате вызова метода ReadingNewsFromRssChanel() класса NewsPublishing.Рассмотрим работу этого класса подробнее.\par


	%-----3.3.1 Класc NewsPublishing----
	\flushleft\addcontentsline{toc}{subsubsection}{3.3.1 Класc NewsPublishing}
	\flushleft\hspace{4ex}\textbf{3.3.1 Класc NewsPublishing}
\flushleft\hspace{4ex}Основная идея класса NewsPublishing заключается в построении списка посредством чтения новостей определённого RSS канала. RSS формат имеет весьма простую структуру. Основными её тегами являются: Title – заголовок статьи, Description – описание статьи, PublishDate – дата публикации, Link – непосредственно ссылка на новость. Список уже прочитанных новостей представляет собой список экземпляров класса News (рис 2), которые содержат информацию о конкретной новости. Помимо этого, в классе хранится список новостей текущей ссылки, сама текущая ссылка на новостную ленту, а также список всех RSS ссылок, которые были считаны с базы данных. Рассмотрим методы данного класса:\par
\begin{itemize}
\item public static IEnumerable<Feed> GetNewsColection();\par
Назначение: возвращает список новостей текущей ссылки – CurrentRssUrl.\par
\item public static void SetRssUrlFromDropListIndex(int index);\par
Назначение: метод устанавливает значение текущей ссылки в зависимости от выбранного индекса всплывающего списка.\par
\item public static void AddUrlToList(string url);\par
Назначение: метод добавляет ссылку на новстную ленту в список всех лент.\par
\item public static void SetListRssUrls(List<String> newListUrls);\par
Назначение: метод утанавливает список ссылок новостных лент.\par
\item public static void RemoveUrlFromList(int index);\par
Наначение: метод удаляет ссылку на новость из списка ссылок всех новостей по выбраному индексу.\par
\item public static void ReadingNewsFromRssChanel();\par
Назначение: данный метод посылает запрос по текущей ссылке – CurrentRssUrl и получает ответ в формате RSS. Считав информацию со всех тегов Item, получаем список новостей данного RSS канала.\par
\end{itemize}
\flushleft\parindent=1cm Серверный c код главной страницы использует следующие классы логику которых рассмотрим подробнее: DriversStandings, TeamStandings.\par
	%-----3.3.2 Класс DriversStandings----
	\addcontentsline{toc}{subsubsection}{3.3.2 Класс DriversStandings}	
	\flushleft\hspace{4ex}\textbf{3.3.2 Класс DriversStandings}
\flushleft\hspace{4ex}Данный класс формирует список пилотов индивидуального зачёта - List<Driver> standingsDrivers, используя метод MakeDriversStandings().\par
 
\flushleft\hspace{4ex}public static void MakeDriversStandings();\par
\hspace{4ex}Назначение: данный метод отправляет запрос по адресу - http://ergast.com/api/f1/2015/driverStandings и получает данные в формате xml (рисунок 3.3.2.1). В документе содержится информация о пилотах данного сезона ( рисунок 3).Прочитав всю необходимую информацию из документа, формируется список List<Driver> standingsDrivers. Другие основные методы:\par
 \begin{itemize}
\item public static string GetCurrentDriverNumber();\par
Назначение: Возвращает номер выбранного гонщика.\par
\item public static IEnumerable<Driver> GetDriverStandings();\par
Назначение: Возвращает список List<Driver> standingsDrivers, так как данный список объявлен с модификатором доступа private.\par
\item private static string GetImageCar(string str);\par
Назначение: Возвращает путь к картинке болида пилота относительно корневого элемента.\par
\item private static string GetImageDriver(string number);\par
Назначение: Возвращает путь к картинке пилота относительно корневого элемента.\par
\end{itemize}
	\includegraphics{DriverStandingsXMLresponse}   
           \center Рисунок 3.3.2.1 – пример Xml ответа, индивидуальный зачёт \par
	%-----3.3.3 Класс TeamStandings----
	\addcontentsline{toc}{subsubsection}{3.3.3 Класс TeamStandings}	
	\flushleft\hspace{4ex}\textbf{3.3.3 Класс TeamStandings}
\flushleft\hspace{4ex}Данный класс формирует список конструкторов - List<Team> standingsTeam, используя метод MakeTeamStandings().\par
\flushleft\hspace{4ex}public static void MakeTeamStandings();\par
\hspace{4ex}Назначение: данный метод отправляет запрос по адресу - http://ergast.com/api/f1/current/constructorStandings и получает данные в формате xml (рисунок 3.3.3.1 ). В документе содержится информация о командах данного сезона.Прочитав всю необходимую информацию из документа, формируется список List<Driver> standingsDrivers. Основные методы:\par
\flushleft\hspace{4ex}public static List<Team> GetTeamStandings();\par`
\hspace{4ex}Назначение: возвращает список конструкторов List<Driver> standingsDrivers.\par	
	\includegraphics{TeamStandingsXMLresponse}   
           \center Рисунок 3.3.3.1 – пример Xml ответа, командный зачёт \par
	%-----3.4 Страница Standings----
	\addcontentsline{toc}{subsection}{3.4 Страница Standings}	
	\flushleft\hspace{4ex}\textbf{3.4 Страница Standings}
\flushleft\hspace{4ex}Данная страница отображает информацию о всех пилотах текущего сезона, а именно: картинка пилота его номер, команду и соответственно имя.
Данная страница генерирует html разметку через компонент Repeater, datasource которого устанавливается на стороне сервера.\par
В случае клика на определённого пилота появляется новая страничка – PersonalDriver.aspx с личной информацией данного пилота: команда, количество подиумов, количество очков, национальность, болид, шлем пилота. Зная номер пилота, можно отобразить информацию о нём.\par
	%-----3.5 Страница Drivers----
	\addcontentsline{toc}{subsection}{3.5 Страница Drivers}	
	\flushleft\hspace{4ex}\textbf{3.5 Страница Drivers}
\flushleft\hspace{4ex}Данная страница отображает информацию о всех пилотах текущего сезона, а именно: картинка пилота его номер, команду и соответственно имя.
Данная страница генерирует html разметку через компонент Repeater, datasource которого устанавливается на стороне сервера.\par
В случае клика на определённого пилота появляется новая страничка – PersonalDriver.aspx с личной информацией данного пилота: команда, количество подиумов, количество очков, национальность, болид, шлем пилота. Зная номер пилота, можно отобразить информацию о нём.\par
	%-----3.6 Страница Schedule----
	\addcontentsline{toc}{subsection}{3.6 Страница Schedule}	
	\flushleft\hspace{4ex}\textbf{3.6 Страница Schedule}
\flushleft\hspace{4ex}Содержимое этой страницы отображает календарь гран-при формулы 1 текущего сезона. В состав содержимого входят: номер гран-при, дата проведения, время проведения, название гран-при и ссылка на информацию в википедии о гран-при. При загрузке страницы срабатывает серверный код, в котором считывается расписание гран-при с сервера и записывается в список класса GrandPrixList и в кэш. Алгоритм представлен на рисунке 3.6.1.\par
	\center\includegraphics{ScheduleSh}   
           \center Рисунок 3.6.1 – событие PageLoad страницы Schedule \par
	%-----3.6.1 Класс GrandPrixList----
	\addcontentsline{toc}{subsubsection}{3.6.1 Класс GrandPrixList}	
	\flushleft\hspace{4ex}\textbf{3.6.1 Класс GrandPrixList}
\flushleft\hspace{4ex}Данный класс формирует расписание сезона формулы 1. Он содержит List<GrandPrix>  с информацией о каждом гран-при, который формируется посредством чтения  Xml документа, полученного из удалённого сервера.
Структура формата Xml представлена на рис 3.6.1.1.\par
	\includegraphics{ScheduleResponseXml}   
           \center Рисунок 3.5.1.1 – пример Xml ответа, расписание гран-при.\par
\flushleft\hspace{4ex}Основные методы класса:\par
\begin{itemize}
\item public static void MakeGrandPrixList();\par
Назначение: считывает расписание гран-при с удалённого сервера в формате xml, записывает всю информацию в список.\par

\item public static List<GrandPrix> GetScheduleList()\par
Назначение: возвращает список гран-при.\par

\item public static void SetScheduleList(List<GrandPrix> grandPrixList);\par
Назначение: устанавливает список гран-при.\par
\end{itemize}

	%-----3.7 Страница LatestRaceResult----
	\addcontentsline{toc}{subsection}{3.7 Страница LatestRaceResult}	
	\flushleft\hspace{4ex}\textbf{3.7 Страница LatestRaceResult}
\flushleft\hspace{4ex}Данная страница отображает последний гран-при, дату проведения, и результаты последнего гран-при: позиция, имя, стартовая позиция, определяемая по результатам квалификации, команда пилота, количество кругов, статус, быстрые круги трассы, время гонки, и количество очков. При загрузке страницы срабатывает серверный код, в котором считываются результаты последней гонки и записываются в список класса LatestRaceInfo и в кэш. Алгоритм представлен на рисунке 3.7.1.2\par
\hspace{4ex}Рассмотрим структуру класса LatestRaceInfo подробнее.\par
	%-----3.7.1 Класс LatestRaceInfo----
	\flushleft\addcontentsline{toc}{subsubsection}{3.7.1 Класс LatestRaceInfo}
	\flushleft\hspace{4ex}\textbf{3.7.1 Класс LatestRaceInfo}
	\flushleft\hspace{4ex}Данный класс формирует список с результатами последней гонки формулы 1. Он содержит List< DriverRaceResult>  с информацией о пилоте, участвовавшем в гран-при.Список формируется посредством чтения  Xml документа, полученного из удалённого сервера. Структура формата Xml представлена на рис 3.7.1.1.
	\includegraphics{ScheduleResponseXml}   
           \center Рисунок 3.7.1.1 – пример Xml ответа, результатов гонки.\par
           \includegraphics{LatestRaceResultSh}   
           \center Рисунок 3.7.1.2 – событие PageLoad страницы LatestRaceResult.\par
           \flushleft\hspace{4ex}Основные методы класса:\par
\begin{itemize}
\item public static List<DriverRaceResult> GetLatestRaceResult();\par
Назначение: возвращает список результата пилотов последнего гран-при.\par
\item public static void SetLatestRaceResult(List<DriverRaceResult> latestRaceRes);\par
Назначение: устанавливает результаты последнего гран-при.\par
\item public static void MakeLatestRaceResult();\par
Назначение: считывает результаты последней гонки с удалённого сервера в формате xml, записывает всю информацию в список.\par
\end{itemize}
           

%-----4 РУКОВОДСТВО ПО УСТАНОВКЕ И ИСПОЛЬЗОВАНИЮ ВЕБ САЙТА------
	\newpage
	\addcontentsline{toc}{section}{4 РУКОВОДСТВО ПОЛЬЗОВАТЕЛЯ}
           \flushleft\hspace{4ex}\section*{\normalsize\hspace{4ex}4 РУКОВОДСТВО ПОЛЬЗОВАТЕЛЯ}
%-----4.1 Развёртывание веб-сайта------
           \addcontentsline{toc}{subsection}{4.1 Развёртывание веб-сайта}
 	\flushleft\hspace{4ex}\section*{\normalsize\hspace{4ex}4.1 Развёртывание веб-сайта}
\flushleft\hspace{4ex}Прежде чем развертывать веб-сайт, нужно подготовить IIS. Главное решение, которое нужно приять, касается места размещения содержимого и его влияния на конечный URL-адрес. Начнем с очевидного подхода - предположим, что необходимо, чтобы URL-адрес для содержимого данного примера был следующим: http://имя сервера:80/Website/F1\par
IIS нужно подготовить так, чтобы было куда скопировать наш файл. В IIS Manager выберите элемент Default Web Site. Как следует из его имени, это сайт по умолчанию на сервере. Щелкните на нем правой кнопкой мыши и в контекстном меню выберите пункт Explore (Проводник), чтобы отрыть окно проводника Windows для заданного по умолчанию каталога IIS, которым является inetpub wwwroot на системном томе (как правило, C:).\par
Создайте каталог Website, а в нем - каталог F1(чтобы обеспечить существование пути inetpub wwwroot-Website-F1). Закройте окно проводника, чтобы вернуться в IIS Manager. Щелкните правой кнопкой на записи Default Web Site и в контекстном меню выберите пункт Refresh (Обновить), чтобы увидеть новый каталог. Переместите файлы веб-сайта на сервер любым подходящим способом - посредством общего сетевого диска, съемного диска USB и т.п. - и скопируйте файлы сайта в каталог F1, созданный на сервере.\par
\hspace{4ex}IIS понадобится также указать, что развернутый сайт является приложением. Это не обязательно, но при развертывании приложений ASP.NET почти всегда будет желательным - активизируется состояние сеанса и другие функциональные средства ASP.NET. Щелкните правой кнопкой мыши на папке F1 в области Connections (Подключения) и в контекстном меню выберите пункт Convert to Application (Преобразовать в приложение). Откроется диалоговое окно Add Application (Добавление приложения). Используемый пул приложений можно изменить, щелкнув на кнопке Select (Выбрать). Настроить учетную запись пользователя, которую IIS будет применять для доступа к содержимому сайта, можно с помощью кнопок Connect as... (Подкл. как...) и Test Settings... (Тест настроек...). Пока что просто щелкните на кнопке ОК. Возможно, придется выбрать пункт Refresh (Обновить) в меню View (Вид) (или, как это часто имеет место, закрыть и снова открыть IIS Manager), но теперь значок записи FileCopy в древовидном представлении должен измениться.\par
%-----4.2 Работа с приложением------
	\addcontentsline{toc}{subsection}{4.2 Работа с приложением}
 	\section*{\normalsize\hspace{4ex}4.2 Работа с приложением}
\flushleft\hspace{4ex}В случае входа на сайт пользователю предоставляется информация о последних гонках, расписании, индивидуальном зачёте пилотов, кубке конструкторов, а также информация о пилотах текущего сезона формулы 1.\par
Кроме того, пользователь может следить за новостями интересующего сайта, просто добавив ссылку на RSS канал в список фидов (рисунок 4.2.1). \par
	\center\includegraphics{OldList}   
           \center Рисунок 4.2.1 – добавление нового фида в список\par
\flushleft\hspace{4ex}Выбрав нужный фид из переченя, пользователю предоставляются последние новости сайта, не посещая его(рисунок 4.2.2).\par
	\center\includegraphics{News}   
           \center Рисунок 4.2.2 –  новости с сайта f1-world.ru\par
\flushleft\hspace{4ex}Также, пользователь может удалять сайт из списка.\par
	%-------ЗАКЛЮЧЕНИЕ-------
	\newpage
	\addcontentsline{toc}{section}{ЗАКЛЮЧЕНИЕ}
	\section*{\center\normalsize ЗАКЛЮЧЕНИЕ \endcenter}
	\flushleft\hspace{4ex}В результате выполнения данной курсовой работы был создан сайт на тему формулы 1 с чтением RSS фидов, который может быть использован в личных целях для отслеживания последних новостей любимых сайтов, не посещая их, что экономит не мало времени на посещение этого сайта и трафика для открытия всей страницы.\par
	\hspace{4ex}В ходе выполнения данной курсовой работы были получены навыки работы с ASP.NET WEB SITE. Детально изучены язык гипертекстовой разметки HTML и таблица каскадных стилей CSS.Получены ознокомительные навыки с базой данных MsSql и средством для работы с базой данных – Entity Framework.\par

\flushleft\hspace{4ex}Возможности сайта:\par
\begin {itemize} 
\item Использование кэша.

\item Считывание данных с удалённого сервера в формате Xml.

\item Работа с базой данных.

\item Отображение новостей интересующего сайта.

\item Добавление и удаление интересующих фидов.
\end {itemize} 

\flushleft\hspace{4ex} При дальнейшем развитие проекта будут добавлены следующие функции:\par

\begin {itemize}
\item Расширенная работа с JavaScript.

\item Оптимизация работы с базой данных.

\item  Возможность просмотра результатов гонок предыдущих сезонов, а также статистики всех пилотов истории формулы 1.
\end{itemize}




	%----СПИСОК ИСПОЛЬЗОВАННОЙ ЛИТЕРАТУРЫ-------
	\newpage
	\addcontentsline{toc}{section}{СПИСОК ИСПОЛЬЗОВАННОЙ ЛИТЕРАТУРЫ}
	\section*{\center\normalsize СПИСОК ИСПОЛЬЗОВАННОЙ ЛИТЕРАТУРЫ \endcenter}
	\begin{enumerate}
	\item Дейтел Х., Дейтел П.. Как программировать на С/С++ – Москва, ООО «И.Д. Вильямс», 2006. – 1454с.
	\item Шилдт Г. С\# 4.0. Полное руководство. – Москва, ООО «И.Д. Вильямс», 2011. – 1056с.
	\item Троелсен Э. Язык программирования С\# 5.0 и платформа .NET 4.5. – Москва, ООО «И.Д. Вильямс», 2013. – 1312с.
	\end{enumerate}






	%----ПРИЛОЖЕНИЕ А (обязательное) Исходный код программы----
	\begin{landscape}
	\newpage
	\addcontentsline{toc}{section}{ПРИЛОЖЕНИЕ А}
	\section*{\center\normalsize ПРИЛОЖЕНИЕ А\\(обязательное)\\Исходный код программы \endcenter}
	\scriptsize
	\begin{lstlisting}
using System;
using System.Collections.Generic;
using System.Linq;
using System.Web;
using System.Web.UI;
using System.Web.UI.WebControls;

public partial class _Default : System.Web.UI.Page
{
    private Repository repository;

    protected IEnumerable<News> GetNewsList()
    {
        return repository.NewsProperty;
    }
    //public List<DriverInfoPilot>;
    protected void Page_Load(object sender, EventArgs e)
    {
        if (!IsPostBack)
        {
            if (DriversStandings.GetDriverStandings() == null)
            {
                DriversStandings.MakeDriversStandings();
                Cache.Insert("DriverStandings",DriversStandings.GetDriverStandings(),null,DateTime.Now.AddMinutes(60),TimeSpan.Zero);                
            }
            else
            {
                    if (GetDriverStandingsFromCache().Count != 0)
                {
                    DriversStandings.SetDriverStandings(Cache["DriverStandings"] as List<Driver>);
                }
                else
                {
                    DriversStandings.MakeDriversStandings();
                    Cache.Insert("DriverStandings", DriversStandings.GetDriverStandings(), null, DateTime.Now.AddMinutes(60), TimeSpan.Zero);
                }               
                
            }
            if (NewsPublishing.GetNewsColection() == null)
            {
                NewsPublishing.SetRssUrlFromDropListIndex(DropDownListRssUrls.SelectedIndex);
                NewsPublishing.ReadingNewsFromRssChanel();
            }
            if (TeamStandings.GetTeamStandings() == null)
            {
                TeamStandings.MakeTeamStandings();
                Cache.Insert("TeamStandings", TeamStandings.GetTeamStandings(), null, DateTime.Now.AddMinutes(60), TimeSpan.Zero);
            }
            else
            {
                if (GetTeamStandingsFromCache().Count != 0)
                {
                    TeamStandings.SetTeamStandings(Cache["TeamStandings"] as List<Team>);
                }
                else
                {
                    TeamStandings.MakeTeamStandings();
                    Cache.Insert("TeamStandings", TeamStandings.GetTeamStandings(), null, DateTime.Now.AddMinutes(60), TimeSpan.Zero);
                }
            }
            if (repository == null)
            {
                repository = new Repository();

                List<string> listRssUrls = new List<string>();
                foreach(News newsClass in GetNewsList())
                {
                    DropDownListRssUrls.Items.Add(newsClass.SiteUrl);
                    listRssUrls.Add(newsClass.RssLinq);
                }
                NewsPublishing.SetListRssUrls(listRssUrls);
            }

            //Cache.Insert("DriverStandings", DriversStandings.GetDriverStandings, null, DateTime.Now.AddMinutes(60), TimeSpan.Zero);
            RepeaterDrivers.DataSource = DriversStandings.GetFirstTenDrivers();
            //RepeaterDrivers.DataSource = DriversStandings.GetDriverStandings();
            RepeaterDrivers.DataBind();
            RepeaterNews.DataSource = NewsPublishing.GetNewsColection();
            RepeaterNews.DataBind();
            RepeaterTeamStandings.DataSource = TeamStandings.GetTeamStandings();
            RepeaterTeamStandings.DataBind();
        }
    }

    protected void ButtonRefresh_Click(object sender, EventArgs e)
    {
        NewsPublishing.ReadingNewsFromRssChanel();
        RepeaterNews.DataSource = NewsPublishing.GetNewsColection();
        RepeaterNews.DataBind();
    }
    protected void DropDownListRssUrls_SelectedIndexChanged(object sender, EventArgs e)
    {
        NewsPublishing.SetRssUrlFromDropListIndex(DropDownListRssUrls.SelectedIndex);
        NewsPublishing.ReadingNewsFromRssChanel();
        RepeaterNews.DataSource = NewsPublishing.GetNewsColection();
        RepeaterNews.DataBind();
    }
    protected void ButtonUrlAdd_Click(object sender, EventArgs e)
    {
        if (Page.IsValid)
        {
            ListItem newElement = new ListItem();
            newElement.Text = TextBoxSite.Text;
            NewsPublishing.AddUrlToList(TextBoxUrl.Text);
            DropDownListRssUrls.Items.Add(newElement);
            using (TestDBContext context = new TestDBContext())
            {
                News news = new News();
                news.RssLinq = TextBoxUrl.Text;
                news.SiteUrl = TextBoxSite.Text;
                context.News.Add(news);
                context.SaveChanges();
            }
        }
    }
    protected void ButtonDelete_Click(object sender, EventArgs e)
    {
        int PrimaryKey = DropDownListRssUrls.SelectedIndex + 1;
        using (TestDBContext context = new TestDBContext())
        {
            context.News.Remove(context.News.Where(n => n.SiteUrl == DropDownListRssUrls.SelectedItem.Text).First());
            context.SaveChanges();
        }
        NewsPublishing.RemoveUrlFromList(DropDownListRssUrls.SelectedIndex);
        DropDownListRssUrls.Items.RemoveAt(DropDownListRssUrls.SelectedIndex);
    }

    private List<Driver> GetDriverStandingsFromCache()
    {
        var ListDriverStandings = Cache["DriverStandings"] as List<Driver>;

        return ListDriverStandings;
    }
    private List<Team> GetTeamStandingsFromCache()
    {
        var listTeam = Cache["TeamStandings"] as List<Team>;
        return listTeam;
    }
}

using System;
using System.Collections.Generic;
using System.Linq;
using System.Web;
using System.Web.UI;
using System.Web.UI.WebControls;

public partial class Standings_Schedule : System.Web.UI.Page
{
    protected void Page_Load(object sender, EventArgs e)
    {
        if(!IsPostBack)
        {
            if(GrandPrixList.GetScheduleList() == null)
            {
                GrandPrixList.MakeGrandPrixList();
                Cache.Insert("GrandPrixList", GrandPrixList.GetScheduleList(), null, DateTime.Now.AddMinutes(60), TimeSpan.Zero);
            }
            else
            {
                if (GetGrandPrixListFromCache().Count != 0)
                {
                    GrandPrixList.SetScheduleList(Cache["GrandPrixList"] as List<GrandPrix>);
                }
                else
                {
                    GrandPrixList.MakeGrandPrixList();
                    Cache.Insert("GrandPrixList", GrandPrixList.GetScheduleList(), null, DateTime.Now.AddMinutes(60), TimeSpan.Zero);
                }
            }
            Season.InnerText = "Season :  " + GrandPrixList.Formula1Season;
            RepeaterSchedules.DataSource = GrandPrixList.GetScheduleList();
            RepeaterSchedules.DataBind();
        }
    }

    private List<GrandPrix> GetGrandPrixListFromCache()
    {
        var grandPrixList = Cache["GrandPrixList"] as List<GrandPrix>;
        return grandPrixList;
    }
}

using System;
using System.Collections.Generic;
using System.Linq;
using System.Web;
using System.Web.UI;
using System.Web.UI.WebControls;
using System.Xml;

public partial class Standings_RaceResultsCurrent : System.Web.UI.Page
{
    protected void Page_Load(object sender, EventArgs e)
    {
        if (!IsPostBack)
        {
            if (DriversStandings.GetDriverStandings() == null)
            {
                DriversStandings.MakeDriversStandings();
            }
            RepeaterDriverStandingsFull.DataSource = DriversStandings.GetDriverStandings();
            RepeaterDriverStandingsFull.DataBind();
        }
    }


    protected void ButtonBackCurrentRace_Click(object sender, EventArgs e)
    {
        Response.Redirect("~/Default.aspx");
    }
}

using System;
using System.Collections.Generic;
using System.Linq;
using System.Web;
using System.Web.UI;
using System.Web.UI.WebControls;

public partial class Standings_PersonalDriver : System.Web.UI.Page
{
    private static Driver curDriver;

    public string CurrentDriverNumber
    {
        get
        {
            string Number;
            Number = Request.QueryString["page"];
            return Number;
        }
    }
    protected void Page_Load(object sender, EventArgs e)
    {
        if (!IsPostBack)
        {
            LabelPersonalDriverNumber.Text = CurrentDriverNumber;
            foreach(Driver curDriver in DriversStandings.GetDriverStandings())
            {
                if (curDriver.PermanentNumber == CurrentDriverNumber)
                {
                    LabelPersonalDriverName.Text = curDriver.NameSurname;
                    LabelPersonalDriverNationality.Text = curDriver.Nationality;
                    LabelPersonalDriverNumber.Text = curDriver.PermanentNumber;
                    LabelPersonalDriverDateOfBirth.Text = curDriver.DateOfBirth;
                    LabelPersonalDriverPodiums.Text = curDriver.Wins;
                    LabelPersonalDriverPoints.Text = curDriver.Points;
                    LabelPersonalDriverTeam.Text = curDriver.Constructor;
                    carImg.Src = curDriver.ImageCar;
                    driverImg.Src = curDriver.ImageDriver;
                    helmetImg.Src = curDriver.ImageHelmet;
                    LabelMainDriverName.Text = curDriver.NameSurname;
                    LabelMainDriverNumber.Text = curDriver.PermanentNumber;
                }
            }
        }
    }
}

using System;
using System.Collections.Generic;
using System.Linq;
using System.Web;
using System.Web.UI;
using System.Web.UI.WebControls;

public partial class Standings_LatestRaceResult : System.Web.UI.Page
{
    protected void Page_Load(object sender, EventArgs e)
    {
        if(!IsPostBack)
        {
            if(LatestRaceInfo.GetLatestRaceResult()==null)
            {
                LatestRaceInfo.MakeLatestRaceResult();
                Cache.Insert("LatestRaceResults", LatestRaceInfo.GetLatestRaceResult(), null, DateTime.Now.AddMinutes(60), TimeSpan.Zero);
            }
            else
            {
                if (GetLatestRaceInfoFromCache().Count != 0)
                {
                    LatestRaceInfo.SetLatestRaceResult(Cache["LatestRaceResults"] as List<LatestRaceInfo.DriverRaceResult>);
                }
                else
                {
                    LatestRaceInfo.MakeLatestRaceResult();
                    Cache.Insert("LatestRaceResults", LatestRaceInfo.GetLatestRaceResult(), null, DateTime.Now.AddMinutes(60), TimeSpan.Zero);
                }
            }
            RaceName.InnerText = LatestRaceInfo.GetGrandPrixName();
            RaceDate.InnerText = "Date :  " + LatestRaceInfo.GetGrandPrixDate();
            RepeaterLastRace.DataSource = LatestRaceInfo.GetLatestRaceResult();
            RepeaterLastRace.DataBind();
        }
    }

    private List<LatestRaceInfo.DriverRaceResult> GetLatestRaceInfoFromCache()
    {
        var latestRaceRes = Cache["LatestRaceResults"] as List<LatestRaceInfo.DriverRaceResult>;
        return latestRaceRes;
    }
}

using System;
using System.Collections.Generic;
using System.Linq;
using System.Web;
using System.Web.UI;
using System.Web.UI.WebControls;
using System.Xml;


public partial class Standings_Standings : System.Web.UI.Page
{
    protected void Page_Load(object sender, EventArgs e)
    {
        if (!IsPostBack)
        {
            if (DriversStandings.GetDriverStandings() == null)
            {
                DriversStandings.MakeDriversStandings();
            }
        }
        RepeaterDriversGeneric.DataSource = DriversStandings.GetDriverStandings();
        RepeaterDriversGeneric.DataBind();       
    }
}

using System;
using System.Collections.Generic;
using System.Linq;
using System.Web;
using System.Xml.Linq;

/// <summary>
/// Сводное описание для NewsPublishing
/// </summary>
public class NewsPublishing
{
    private static List<Feed> NewsCollection;
    private static string CurrentRssUrl {get;set;}

    //private static List<string> listRssUrls = new List<string> { "http://www.f1-world.ru/news/rssexp6.xml", "http://www.f1news.ru/export/news.xml", "http://www.championat.com/xml/rss_auto_f1-article.xml", "http://www.autofaq.com.ua/rss/blog/formula-1/" };
    private static List<string> listRssUrls;
    public static IEnumerable<Feed> GetNewsColection()
    {
        return NewsCollection;
    }

    public static void SetRssUrlFromDropListIndex(int index)
    {
        if (index != -1)
        {
            CurrentRssUrl = listRssUrls.ElementAt(index);
        }
        else
            CurrentRssUrl = null;
    }

    public static void AddUrlToList(string url)
    {
        listRssUrls.Add(url);
    }

    public static void SetListRssUrls(List<String> newListUrls)
    {
        listRssUrls = newListUrls;
    }

    public static void RemoveUrlFromList(int index)
    {
        listRssUrls.RemoveAt(index);
    }
    public static void ReadingNewsFromRssChanel()
    {
        string rssFeedUrl = CurrentRssUrl;
        NewsCollection = new List<Feed>();
        try
        {
            XDocument xDoc = new XDocument();
            xDoc = XDocument.Load(rssFeedUrl);

            var k = xDoc.Descendants("item");
          
            var items = (from x in xDoc.Descendants("item")
                         select new
                         {
                             title = x.Element("title").Value,
                             link = x.Element("link").Value,
                             pubDate = x.Element("pubDate").Value,
                             description = x.Element("description").Value,
                         });
            if (items != null)
            {
                foreach (var i in items)
                {
                    Feed feed = new Feed
                    {
                        Title = i.title,
                        Link = i.link,
                        PublishDate = i.pubDate,
                        Description = i.description,
                    };
                    NewsCollection.Add(feed);
                }
            }
        }
        catch (Exception ex)
        {
            NewsCollection.Clear();
        }
    }
}

using System;
using System.Collections.Generic;
using System.Linq;
using System.Web;

/// <summary>
/// Сводное описание для Repository
/// </summary>
public class Repository
{
    private TestDBContext context = new TestDBContext();

    public IEnumerable<News> NewsProperty
    {
        get { return context.News; }
    }
}

using System;
using System.Collections.Generic;
using System.Linq;
using System.Web;
using System.Xml;

/// <summary>
/// Сводное описание для TeamStandings
/// </summary>
public class TeamStandings
{
    private static List<Team> standingsTeam;

    public static List<Team> GetTeamStandings()
    {
        return standingsTeam;
    }

    public static void SetTeamStandings(List<Team> teamStandings)
    {
        standingsTeam = teamStandings;
    }

    public static void MakeTeamStandings()
    {
        string teamStandingsUrl = "http://ergast.com/api/f1/current/constructorStandings";
        standingsTeam = new List<Team>();
        try
        {
            XmlDocument xDoc = new XmlDocument();
            xDoc.Load(teamStandingsUrl);

            var i = xDoc.LastChild.FirstChild.ChildNodes;

            foreach (XmlElement xmlelement in xDoc.LastChild.FirstChild.FirstChild.ChildNodes)
            {
                Team team = new Team();

                team.Position = xmlelement.Attributes[0].Value;
                team.Points = xmlelement.Attributes[2].Value;
                team.Wins = xmlelement.Attributes[3].Value;

                team.Name = xmlelement.FirstChild.ChildNodes[0].InnerText;
                team.Nationality = xmlelement.FirstChild.ChildNodes[1].InnerText;

                standingsTeam.Add(team);
            }
        }
        catch (Exception ex)
        {
            standingsTeam.Clear();
        }
    }
}

using System;
using System.Collections.Generic;
using System.Linq;
using System.Web;

/// <summary>
/// Сводное описание для Team
/// </summary>
public class Team
{
    public string Position { get; set; }
    public string Points { get; set; }
    public string Wins { get; set; }
    public string Name { get; set; }
    public string Nationality { get; set; }
}

using System;
using System.Collections.Generic;
using System.Linq;
using System.Web;
using System.Data.Entity;
using System.ComponentModel.DataAnnotations;
using System.ComponentModel.DataAnnotations.Schema;

/// <summary>
/// Сводное описание для NewsClass
/// </summary>

public class NewsClass
{
    public int NewsID { get; set; }
    public string SiteUrl { get; set; }
    public string RssLinq { get; set; }
}

using System;
using System.Collections.Generic;
using System.Linq;
using System.Web;
using System.Xml;

/// <summary>
/// Сводное описание для LatestRaceInfo
/// </summary>
public class LatestRaceInfo
{
    public class DriverRaceResult
    {
        public string Name {get;set;}
        public string Surname {get;set;}
        public string NameSurname {get;set;}
        public string Constructor {get;set;}
        public string Grid {get;set;}
        public string Laps {get;set;}
        public string Status {get;set;}
        public string Position {get;set;}
        public string Points {get;set;}
        public string Time {get;set;}
        public string FastestLap{get;set;}
    }

    private static string GrandPrixName { get; set; }
    private static string GrandPrixDate {get;set;}
    public static string GetGrandPrixName()
    {
        return GrandPrixName;
    }
    public static string GetGrandPrixDate()
    {
        return GrandPrixDate;
    }

    private static List<DriverRaceResult> LatestRaceResults;

    public static List<DriverRaceResult> GetLatestRaceResult()
    {
        return LatestRaceResults;
    }

    public static void SetLatestRaceResult(List<DriverRaceResult> latestRaceRes)
    {
        LatestRaceResults = latestRaceRes;
    }
	
    public static void MakeLatestRaceResult()
    {
        string lastRaceSource = "http://ergast.com/api/f1/current/last/results";
        LatestRaceResults = new List<DriverRaceResult>();
        try
        {

            XmlDocument xDoc = new XmlDocument();
            xDoc.Load(lastRaceSource);

            GrandPrixName = xDoc.LastChild.FirstChild.FirstChild.ChildNodes[0].InnerText;
            GrandPrixDate = xDoc.LastChild.FirstChild.FirstChild.ChildNodes[2].InnerText;

            foreach (XmlElement xmlelement in xDoc.LastChild.FirstChild.FirstChild.ChildNodes[4])
            {
                Team team = new Team();

                team.Position = xmlelement.Attributes[0].Value;
                team.Points = xmlelement.Attributes[2].Value;
                team.Points = xmlelement.Attributes[3].Value;

                team.Name = xmlelement.FirstChild.ChildNodes[0].InnerText;
                team.Nationality = xmlelement.FirstChild.ChildNodes[1].InnerText;

                DriverRaceResult driver = new DriverRaceResult();

                driver.Points = xmlelement.Attributes[3].Value;
                driver.Position = xmlelement.Attributes[1].Value;

                driver.Name = xmlelement.ChildNodes[0].ChildNodes[1].InnerText;
                driver.Surname = xmlelement.ChildNodes[0].ChildNodes[2].InnerText;
                driver.NameSurname = driver.Name + " " + driver.Surname; 
                driver.Constructor = xmlelement.ChildNodes[1].ChildNodes[0].InnerText;
                driver.Grid = xmlelement.ChildNodes[2].InnerText;
                driver.Laps = xmlelement.ChildNodes[3].InnerText;
                driver.Status = xmlelement.ChildNodes[4].FirstChild.InnerText;
                if (xmlelement.ChildNodes[5] == null)
                {
                    driver.Time = "";
                }
                else
                {
                    driver.Time = xmlelement.ChildNodes[5].InnerText;
                }
                if (xmlelement.ChildNodes[6] == null)
                {
                    driver.FastestLap = "";
                }
                else
                {
                    driver.FastestLap = xmlelement.ChildNodes[6].ChildNodes[0].InnerText;
                }
                LatestRaceResults.Add(driver);
            }
        }
        catch (Exception ex)
        {
            LatestRaceResults.Clear();
        }
    }
}

using System;
using System.Collections.Generic;
using System.Linq;
using System.Web;
using System.Xml;

/// <summary>
/// Сводное описание для GrandPrixList
/// </summary>
public class GrandPrixList
{
    private static List<GrandPrix> scheduleList;

	public static string Formula1Season;
    public static void MakeGrandPrixList()
    {
        string Url = "http://ergast.com/api/f1/current";
        try
        {
            scheduleList = new List<GrandPrix>();
            XmlDocument xDoc = new XmlDocument();
            xDoc.Load(Url);

            Formula1Season = xDoc.LastChild.FirstChild.Attributes[0].Value;

            foreach(XmlElement race in xDoc.LastChild.FirstChild.ChildNodes)
            {
                GrandPrix grandPrix = new GrandPrix();
                grandPrix.RaceName = race.ChildNodes[0].InnerText;
                grandPrix.Date = race.ChildNodes[2].InnerText;
                grandPrix.Time = race.ChildNodes[3].InnerText;
                grandPrix.Round = race.Attributes[1].InnerText;
                grandPrix.Url = race.Attributes[2].InnerText;
                scheduleList.Add(grandPrix);
            }
        }
        catch (Exception ex)
        {
            scheduleList.Clear();
        }
    }
    public static List<GrandPrix> GetScheduleList()
    {
        return scheduleList;
    }
    public static void SetScheduleList(List<GrandPrix> grandPrixList)
    {
        scheduleList = grandPrixList;
    }
}

using System;
using System.Collections.Generic;
using System.Linq;
using System.Web;

/// <summary>
/// Сводное описание для GrandPrix
/// </summary>
public class GrandPrix
{
    public string RaceName { get; set; }
    public string Date { get; set; }
    public string Time { get; set; }
    public string Round { get; set; }
    public string Url { get; set; }
}

using System;
using System.Collections.Generic;
using System.Linq;
using System.Web;

/// <summary>
/// Сводное описание для Feed
/// </summary>
public class Feed
{
        public string Title { get; set; }
        public string Link { get; set; }
        public string Description { get; set; }
        public string PublishDate { get; set; }
}

using System;
using System.Collections.Generic;
using System.Linq;
using System.Web;
using System.Xml;

/// <summary>
/// Сводное описание для DriversStandings
/// </summary>
public class DriversStandings
{

    private static List<Driver> standingsDrivers;
    private static string CurrentDriverNumber { get; set; }
    public static void SetCurrentDriverNumber(string number)
    {
        CurrentDriverNumber = number;
    }

    public static List<Driver> GetFirstTenDrivers()
    {
        List<Driver> topTenList = new List<Driver>();
        int count = 0;
        try
        {
            while (count < 10)
            {
                topTenList.Add(standingsDrivers[count]);
                count++;
            }
        }
        catch(Exception ex)
        {
            topTenList.Clear();
        }
        return topTenList;
    }

    public static string GetCurrentDriverNumber()
    {
        return CurrentDriverNumber;
    }
    public static List<Driver> GetDriverStandings()
    {
        return standingsDrivers;
    }
    public static void SetDriverStandings(List<Driver> listDriverStandings)
    {
        standingsDrivers = listDriverStandings;
    }
    //public static IEnumerable<Driver> GetBest10DriverStandings()
    //{
        
    //}
    public static void MakeDriversStandings()
    {
        string driverStandingsUrl = "http://ergast.com/api/f1/2015/driverStandings";
        try
        {
            standingsDrivers = new List<Driver>();
            XmlDocument xDoc = new XmlDocument();
            xDoc.Load(driverStandingsUrl);

            foreach (XmlElement xmlelement in xDoc.LastChild.FirstChild.FirstChild.ChildNodes)
            {
                Driver driver = new Driver();
                driver.Position = xmlelement.Attributes[0].Value;
                driver.Points = xmlelement.Attributes[2].Value;
                driver.Wins = xmlelement.Attributes[3].Value;

                driver.PermanentNumber = xmlelement.FirstChild.ChildNodes[0].InnerText;
                driver.ImageDriver = GetImageDriver(driver.PermanentNumber);
                driver.GivenName = xmlelement.FirstChild.ChildNodes[1].InnerText;
                driver.FamilyName = xmlelement.FirstChild.ChildNodes[2].InnerText;
                driver.DateOfBirth = xmlelement.FirstChild.ChildNodes[3].InnerText;
                driver.Nationality = xmlelement.FirstChild.ChildNodes[4].InnerText;

                driver.Constructor = xmlelement.LastChild.ChildNodes[0].InnerText;
                driver.ImageCar = GetImageCar(driver.Constructor);
                driver.ImageHelmet = GetDriverHelmet(driver.PermanentNumber);
                driver.NameSurname = driver.GivenName + " " + driver.FamilyName;
                if (driver.PermanentNumber != "20")
                {
                    standingsDrivers.Add(driver);
                }
            }
        }
        catch (Exception ex)
        {
            standingsDrivers.Clear();
        }
    }

    private static string GetImageCar(string strTeam)
    {
        switch (strTeam)
        {
            case "Mercedes":
                return "../Images/teamCarMercedes.jpg";
            case "Ferrari":
                return "../Images/teamCarFerrari.jpg";
            case "Williams":
                return "../Images/teamCarWilliams.jpg";
            case "Lotus F1":
                return "../Images/teamCarLotus.jpg";
            case "Sauber":
                return "../Images/teamCarSauber.jpg";
            case "Red Bull":
                return "../Images/teamCarRedBull.jpg";
            case "Force India":
                return "../Images/teamCarForceIndia.jpg";
            case "Toro Rosso":
                return "../Images/teamCarToroRosso.jpg";
            case "McLaren":
                return "../Images/teamCarMcLaren.jpg";
            default:
                return "../Images/default.jpg";
        }
    }

    private static string GetImageDriver(string number)
    {
        string pathImage = "../ImagesDrivers/";
        string fullPath;
        fullPath = String.Concat(pathImage, Convert.ToString(number), ".jpg");

        return fullPath;
    }

    private static string GetDriverHelmet(string number)
    {
        string fullpath = "../Images/helmet.png";
        return fullpath;
    }
}   

using System;
using System.Collections.Generic;
using System.Linq;
using System.Web;

/// <summary>
/// Сводное описание для Driver
/// </summary>
public class Driver
{
    public string Position { get; set; }
    public string Points { get; set; }
    public string Wins { get; set; }
    public string PermanentNumber { get; set; }
    public string GivenName { get; set; }
    public string FamilyName { get; set; }
    public string Nationality { get; set; }
    public string Constructor { get; set; }
    public string DateOfBirth { get; set; }
    public string ImageCar { get; set; }
    public string ImageDriver { get; set; }
    public string NameSurname { get; set; }
    public string ImageHelmet { get; set; }
}	
	
	\end{lstlisting}
	\end{landscape}
	
	
\end{document}          
